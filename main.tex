%! TeX program = lualatex
%! TEX options = -synctex=1 -interaction=nonstopmode -file-line-error --shell-esca    pe "%DOC%"
\documentclass[a5paper,10pt]{article}

% \usepackage[T1,T2A]{fontenc}
\usepackage[lutf8]{luainputenc}
\usepackage{graphicx} 
\usepackage{enumitem}
\usepackage{indentfirst}
\usepackage{titlesec}
\usepackage{xcolor}
\usepackage{afterpage}
\usepackage[autostyle]{csquotes}
\usepackage[russian]{babel}
\usepackage[all]{nowidow}               % usuń pojedyńcze zdania na stronie (wdowy)
\usepackage[nosingleletter]{impnattypo} % usuń pojednyncze litery (sierotki)
% % poniższe 4 linijki zapobiegają łamaniu się słów
\tolerance=1
\emergencystretch=\maxdimen
\hyphenpenalty=10000
\hbadness=10000

% \AtBeginEnvironment{quote}{\singlespacing\small}
% \DeclareQuoteAlias{dutch}{polish}

\newcommand\blankpage{%
    \null
    \thispagestyle{empty}%
    \addtocounter{page}{-1}%
    \newpage}

% Marginsesy 
\usepackage[
    twoside,
    left=2cm,
    top=1.5cm,
    right=2cm,
    bottom=1.5cm,
    bindingoffset=0cm]{geometry} 
    
% Pod spodem masz fajną czcionkę - wystarczy że będziesz dawał tą 
% linijkę na początku dokumentu i będziesz kompliować nie w pdfLATEX a w LuaLatex
\usepackage{libertine}  

%%%%%%%%%%%%%%%%%%%%%%%%%%%%%%%%%%%%%%%%%%%%%%%%%%%%%%%%
\usepackage{fancyhdr}

\pagestyle{fancy}
\fancyhf{}
\renewcommand{\headrulewidth}{0pt}
\fancyfoot[RO]{\textcolor{gray75}{|} \thepage}
\fancyfoot[LE]{\thepage~\textcolor{gray75}{|}}
%%%%%%%%%%%%%%%%%%%%%%%%%%%%%%%%%%%%%%%%%%%%%%%%%%%%%%%%

\setlength{\parindent}{1.8em}
\setlength{\parskip}{.5em}
\renewcommand{\baselinestretch}{1.2}

\definecolor{gray75}{gray}{0.75}
\newcommand{\hsp}{\hspace{5pt}}
\titleformat{\section}[hang]
{\Large\bfseries}{\thesection\hsp\textcolor{gray75}{|}\hsp}{0pt}{\Large\bfseries\scshape}

\title{\huge{Fides et ratio}}
\author{Święty Jan Paweł II}
\date{14 września 1998}


\begin{document}

%%%%%%%%%%%%%%
% Title-page %
%%%%%%%%%%%%%%

% \afterpage{\blankpage}

\makeatletter
    \begin{titlepage}
        \begin{center}
            \vspace*{0.5cm} % Większe przerwy
            {\huge \scshape  \bfseries  \@title }\\[2ex] 
            {\large  \@author}\\[4ex]
            \vspace*{1.5cm}
            % \includegraphics[width=0.7\paperwidth]{josemaria.jpg} \\[4ex]
            \vfill
            {\normalsize \@date}
        \end{center}
    \end{titlepage}
\makeatother

%%%%%%%%%%%%%%%
% Second page %
%%%%%%%%%%%%%%%

\newpage

\pagestyle{empty}

 \vspace*{\fill}

% \noindent\hrulefill

% {\footnotesize 
%     \noindent Tu można dać info skąd wziąłeś tekst 
    
%     \medskip
    
% 	\noindent opracowanie: K. Pyziołek\\
% 	skład i łamanie: M. Siemaszko
	
% 	\medskip

% 	\noindent email: karus.pyziol@gmail.com
% }

\newpage

%%%%%%%%%%%%
% Document %
%%%%%%%%%%%%

\setcounter{page}{1} %Start the actually document on page 1
\pagestyle{fancy}
«Познай самого себя»

1. Обращаясь к истории как Востока, так и Запада, можно заметить, что в течение
веков человек проделал определенный путь, который вел его к постепенному
познанию истины и встрече с ней. Этот процесс происходил — иначе и быть не
могло — в сфере самопознания личности. Чем больше человек постигает
действительность и мир, тем в большей степени он осознает себя как единственное
в своем роде существо, а вместе с тем для него все более насущным становится
вопрос о смысле вещей и своего собственного существования. Все, что является
объектом нашего познания, становится частью нашей жизни. Призыв Gnothi seauton
(«познай самого себя»), начертанный на архитраве храма в Дельфах, — это
свидетельство фундаментальной истины, которую должен считать элементарным
исходным правилом каждый человек, который, желая отличаться от прочих творений,
называет себя «человеком», то есть «познающим самого себя».

Внимательный взгляд на древнюю историю в изобилии находит свидетельства тому,
что в различных частях света, где развивались самобытные культуры, люди
одновременно задавали себе вопросы, которыми знаменуется жизнь всех людей: «Кто
я? Откуда я пришел? Куда я иду? Почему существует зло? Что ждет меня после этой
жизни?» Мы находим эти вопросы в святых письменах Израиля, а также в Ведах и
«Авесте»; мы встречаем их в трудах Конфуция и Лао-Цзы, в проповедях тиртханкар
и самого Будды, в поэмах Гомера и в трагедиях Еврипида и Софокла, в философских
сочинениях Платона и Аристотеля. Все эти вопросы вызваны необходимостью обрести
смысл (потребность в котором человек изначально очень остро ощущает в своем
сердце), и от ответа на эти вопросы зависит, какое направление человек изберет
в своей жизни.

2. Эти поиски не чужды и не могут быть чужды Церкви. С того момента, когда в
Пасхальной тайне Церковь получила в дар совершенную истину о жизни человека,
она странствует по земле, проповедуя, что Иисус Христос — это «путь, истина и
жизнь» (Ин 14,6). Среди различных форм служения Церкви на благо человека одну
она считает всецело себе свойственной: служение истине \footnote{Я уже писал об
    этом в моей первой энциклике Redemptor hominis: «Мы стали соучастниками
    миссии Христа-Пророка и в силу этой миссии вместес Ним служим в Церкви
    Божественной истине. Быть ответственным за истину — значит также любить ее
    и стремиться к более точному ее пониманию,с тем чтобы мы теснее
    приблизились к ней во всей ее спасительной силе,во всем ее великолепии и во
всей ее глубине и простоте». П. 19: AAS 71 (1979),306.} . С одной стороны, в
этом служении верующие делаются общниками тех усилий, которые люди направляют
на поиск истины \footnote{См. Второй Ватиканский Собор, Пастырская конституция
о Церкви в современном мире Gaudium et spes, 16.} , а с другой — они обязаны
делиться познанными ими достоверными истинами, сознавая, однако, что любая
познанная истина является лишь ступенью на пути к познанию той истины, которую
Бог раскроет в окончательном откровении: «Теперь мы видим как бы сквозь тусклое
стекло, гадательно, тогда же лицом к лицу; теперь знаю я отчасти, а тогда
познаю, подобно как я познан» (1 Кор 13, 12).

3. Существует много средств, при помощи которых человек может устремляться к
более совершенному познанию истины и благодаря этому делать свою жизнь все
более соответствующей человеческому достоинству. Среди них особое место
занимает философия, которая прямо ставит вопросы о смысле жизни и предугадывает
на

них ответ, а следовательно, является одной из самых возвышенных форм служения
человека. Термин «философия» греческого происхождения и означает «любовь к
мудрости». Действительно, философия возникла и начала развиваться в эпоху,
когда человек стал задаваться вопросами о причинах и целях вещей. Используя
различные формы и средства, она показывает, что жажда истины свойственна самой
природе человека. Врожденным свойством человеческого разума является склонность
к размышлениям о причинах вещей, хотя ответы, которые он со временем давал на
возникавшие вопросы, однозначно свидетельствуют о взаимодействии различных
культур, в условиях которых живут люди.

Философия оказала сильное влияние на формирование и развитие культуры на
Западе, но мы также не должны забывать о ее воздействии на представления о
человеческой жизни, распространенные на Востоке. Каждый народ хранит свою
исконную оригинальную мудрость, которая является сокровищем культуры и
стремится воплотиться, прежде всего, в философских формах. Правильность этого
утверждения подтверждается тем, что некая основная форма философских знаний и в
наше время может быть обнаружена в тех постулатах, которые послужили основой
для создания законодательств разных стран и международного права, регулирующих
нормы общественной жизни.

4. Как бы то ни было, необходимо помнить, что под одним именем может быть
сокрыто несколько значений. Поэтому здесь следует сделать предварительное
разъяснение. Желая открыть окончательную истину о своем существовании, человек
старается приобрести те универсальные знания, которые позволят ему лучше
познать самого себя и добиться большего самоусовершенствования. Эти основные
знания рождаются из того удивления, которое пробуждается в нем от созерцания
мироздания: человек с изумлением замечает, что живет в мире и связан с другими
подобными существами общим предназначением. Именно в этот момент он начинает
путь, по которому будет следовать к открытию все новых горизонтов знаний. Если
бы человек не удивлялся, он был бы поглощен рутиной, перестал бы развиваться и
вскоре не смог бы вести подлинно личностную жизнь.

Способность к абстрагированию, свойственная человеческому разуму, позволяет ему
придать точную форму своему мышлению с помощью философских навыков и таким
образом выработать систему знаний, отличающуюся логически взаимосогласованными
утверждениями и гармоничной связью отдельных дисциплин. Благодаря этому
процессу в различных культурах и в разные эпохи были получены результаты,
которые позволили создать целостные системы мышления. На практике это часто
приводило к соблазну отождествить одно избранное направление со всей
философией. Однако очевидно, что в таких случаях рождается своего рода
«философская гордыня», которая побуждает человека придавать своему
несовершенному видению мира, ограниченному выбором определенной точки зрения,
статус универсальной интерпретации. В действительности каждая философская
система, хотя и заслуживает безусловного уважения как нечто целое и
всеохватывающее, должна признавать первенство философского мышления, из
которого она рождается и которому обязана служить.

Исходя из этого, можно выделить те философские истины, которые, несмотря на ход
времени и совершенствование знаний, продолжают оставаться актуальными. К их
числу можно отнести, например, принципы противоречия, целесообразности и
причинности, концепцию личности как свободного и разумного субъекта и ее
способность постичь Бога, истину и добро; к ним также относятся основные
моральные принципы, признанные всеми. Эти и другие аргументы доказывают, что,
несмотря на различные школы мышления, существует определенный свод знаний,
которые можно считать неким духовным наследием человечества. Мы видим здесь как
бы имплицитную философию, знание принципов которой ощущает в себе каждый
человек, пусть даже в общих чертах и неосознанно. Именно потому, что эти
принципы в определенной степени известны всем, они должны являться отправной
точкой для различных философских школ. Если разум сможет интуитивно найти и
сформулировать начальные и универсальные жизненные принципы и сделать
правильные логические и онтологические выводы, его можно назвать правым
разумом, или, как говорили древние, orthos logos.

5. В свою очередь, Церковь высоко ценит стремление разума к целям, благодаря
которым сама жизнь людей делается более достойной. Она видит в философии путь к
постижению основных истин о жизни человека. Одновременно она считает философию
необходимым инструментом, который помогает глубже познать веру и передать
евангельские истины тем, кто их еще не знает.

Вспоминая инициативы моих предшественников, я также хочу обратить внимание на
эту специфическую активность разума. К этому меня побуждает убеждение, что в
наше время поиск совершенной истины часто представляется особенно затрудненным.
Несомненно, заслугой современной философии является то, что она сосредоточила
свое внимание на человеке. Исходя из этого и пытаясь решить многочисленные
вопросы, разум еще отчетливее осознал необходимость получать более широкие и
глубокие познания. В результате были созданы сложные системы мышления, которые
вызвали развитие различных областей знаний, способствуя прогрессу культуры и
цивилизации. Антропология, логика, естественные науки, история, лингвистика...
В определенном отношении этот процесс охватил все области знаний. Но, несмотря
на положительные достижения, следует помнить: тот же разум, сосредоточенный на
односторонних поисках знаний о человеке как субъекте, похоже, совершенно
забывает, что человек всегда призван стремиться к истине, которая
трансцендентна ему самому. Без этого постоянного стремления к трансцендентной
истине каждый человек зависит от произвола своих собственных суждений, а его
существование как личности оценивается исключительно с помощью прагматических
критериев, которые по своей природе основываются на экспериментальных данных
вследствие ошибочного убеждения, что все должно быть подчинено технике. В
результате, вместо того чтобы стремиться к истине, разум под бременем столь
обширных познаний обращается к самому себе, постепенно лишается способности
восходить к высшей реальности и уже не дерзает искать истинный смысл
существования. Современная философия забыла, что именно бытие должно являться
предметом ее изучения, и ограничилась познанием человека. Вместо того чтобы
использовать данную человеку способность познавать истину, она предпочитает
подчеркивать его ограниченность и зависимость от внешних условий.

Это привело к возникновению различных видов агностицизма и релятивизма, и в
результате философские искания увязли в трясине всеобщего скептицизма. В
последнее время распространяются различные доктрины, которые пытаются
подвергнуть сомнению даже ценность тех истин, которые человек уже считал своим
приобретением. Законное многообразие мнений сменил аморфный плюрализм,
основанный на предпосылке, что все мнения имеют одинаковую ценность. Это одно
из наиболее распространенных проявлений сомнения в существовании истины,
которое наблюдается в современном мире. Такая позиция встречается также в
некоторых представлениях о жизни, которые родились на Востоке; они лишают
истину абсолютного характера, исходя из того, что она в равной степени
присутствует в различных, даже противоречащих друг другу учениях. В такой
перспективе все сводится к частному мнению. Можно сказать, что мы наблюдаем
движение, которое не имеет постоянного направления: с одной стороны,
философские размышления следуют по пути, который приближает их к жизни людей и
формам, в которых она выражается, а с другой — философы предпочитают
рассматривать экзистенциальные, герменевтические или языковые проблемы, которые
не затрагивают основного вопроса об истине жизни каждого человека, о бытии и о
Самом Боге. Вследствие этого современные люди (не только философы) с недоверием
относятся к познавательным способностям человека. Из ложной скромности человек
довольствуется неполными и преходящими истинами и уже не пытается ставить
извечных вопросов о смысле и первооснове жизни человека и общества. Короче,
можно сказать, что им уже утрачена надежда получить от философии окончательные
ответы на эти вопросы.

6. Церковь, в силу своего авторитета хранительницы Откровения Иисуса Христа,
хочет подтвердить необходимость размышления об истине. Поэтому я решил
обратиться к Вам, досточтимые братья во епископате, с которыми меня связывает
миссия «являть истину» (2 Кор 4, 2), к богословам и философам, которые обязаны
исследовать различные аспекты истины, а также ко всем людям, ищущим истину, и
поделиться размышлениями о стремлении к истинной мудрости, чтобы каждый, кто
хранит любовь к ней в сердце своем, мог выбрать правильный путь, позволяющий
обрести ее и найти в ней упокоение от трудов своих и духовную радость.

К этому меня прежде всего склоняет мысль, которую выразил Второй Ватиканский
Собор: епископы являются «свидетелями Божественной и католической истины»
\footnote{Догматическая конституция о Церкви Lumen gentium, 25.} .
Следовательно, нам, епископам, поручено проповедовать истину; мы не можем
уклоняться от этой задачи, поскольку иначе мы бы оставили принятое нами
служение. Подтверждая истины веры, мы можем вернуть современному человеку веру
в свои способности к познанию и одновременно призвать философию заново обрести
и осознать свое достоинство.

Еще одно обстоятельство вынуждает меня представить эти размышления. В энциклике
Veritatis splendor я обратил внимание на «некоторые основные истины
католической доктрины в связи с наблюдаемыми в последнее время стремлениями
отвергнуть их или исказить» \footnote{П. 4: AAS 85 (1993), 1136.}. В настоящей
энциклике я бы хотел продолжить эту мысль и сосредоточиться на понятии самой
истины и ее основе по отношению к вере, ибо несомненно, что прежде всего
молодое поколение, от которого зависит и которому принадлежит будущее, в эпоху
стремительных и сложных перемен может потерять истинные ориентиры.
Необходимость обрести прочный фундамент, на котором можно построить личную и
общественную жизнь, особенно остро ощущается тогда, когда человеку требуется
показать, насколько несовершенны те суждения, которые возводят временную и
преходящую действительность в ранг важнейших ценностей, порождая тем самым
ложные надежды на открытие истинного смысла жизни. Вот почему многие люди
оказываются на краю пропасти, не понимая, что ждет их дальше. Это происходит
также и потому, что часто те, кто призван выражать плоды своих размышлений в
различных формах культуры, отворачиваются от истины, выше ценя преходящий
успех, нежели терпеливые поиски подлинного содержания жизни. Поэтому философия
обязана обрести свое прежнее призвание, ибо на нее возложена обязанность
формировать человеческую мысль и культуру и неустанно призывать людей к поиску
истины. Именно поэтому я чувствую не только необходимость, но и моральную
обязанность высказаться на эту тему, чтобы на пороге третьего тысячелетия
христианской эры человечество четко осознало, какие способности оно получило в
дар, и вновь начало мужественно осуществлять план спасения, в который включена
сама его история. 

\section{Иисус, являющий Отца}

7. В основе любых рассуждений, проводимых Церковью, лежит убеждение, что ей
доверено учение, которое исходит от самого Бога (см. 2 Кор 4, 1-2). Не из
самостоятельных размышлений, пусть даже самых возвышенных, Церковь получила то
знание, которое она дает людям, но из слова Божьего, принятого с верой (см. 1
Фес2, 13). Источником нашей веры является единственная в своем роде встреча,
которая означает открытие многовековой тайны (см. 1 Кор2, 7; Рим 16, 25-26):
«Было благоугодно Богу в Его благости и премудрости явить Самого Себя и
поведать тайну Своей воли (см. Еф1, 9), благодаря которой люди через Христа,
воплотившееся Слово, в Духе Святом имеют доступ к Отцу и становятся
причастниками Божественного естества» \footnote{Второй Ватиканский Собор,
Догматическая конституция о БожественномОткровении Dei Verbum, 2.} . Это
совершенно бескорыстная инициатива, которая исходит от Бога, чтобы затронуть
людей и спасти их. Будучи источником любви, Бог хочет, чтобы Его постигли, а
познание Бога человеком приводит к совершенству всякое иное знание, к которому
способен прийти человек касательно смысла собственного существования.

8. Почти дословно переняв учение, содержащееся в конституции Dei Filius Первого
Ватиканского Собора, и учитывая принципы, сформулированные на Тридентском
Соборе, отцы Второго Ва-

тиканского Собора конституцией Dei Verbum продолжили многовековой путь
осознания веры с помощью размышлений об Откровении в свете библейского учения и
традиций патристики. Участники Первого Ватиканского Собора подчеркнули
сверхъестественный характер Божественного Откровения. Рационалистическая
критика, которая в тот период объявила поход против веры, основываясь на
ложных, но широко распространенных тезисах, отрицала всякое познание, которое
не является плодом природных способностей разума. В связи с этим собор
подтвердил истину, что наряду с познанием, свойственным человеческому разуму,
который в силу своей природы может достичь Самого Творца, существует познание,
свойственное только вере. Это познание выражает истину, основанную на
откровении Бога о Самом Себе, и эта истина является достовернейшей, ибо Бог не
заблуждается и не желает вводить в заблуждение \footnote{См. Догматическая конституция о католической вере Dei Filius, III, DS 3008.} .

9. Итак, Первый Ватиканский Собор учит, что истина, постигаемая в результате
философских размышлений, и истина Откровения не повторяют друг друга, и ни одна
из них не делает другую излишней: «Существуют два вида познания, которые имеют
не только разные источники, но и объекты. У них разные источники, поскольку в
первом случае мы познаем с помощью естественного разума, а во втором — с
помощью веры. У них разные объекты, поскольку кроме истины, к познанию которой
может прийти естественный разум, нам предлагается верить в тайны, скрытые в
Боге: их невозможно постичь без Божественного Откровения» \footnote{Там же,
глава IV: DS 3015; цитируется также в: Второй Ватиканский Собор,Пастырская
конституция Gaudium et spes, 59.} . Вера, основывающаяся на свидетельстве
Божием и укрепляемая сверхъестественной благодатью, в действительности
относится к иному порядку, нежели философское познание. Последнее основано на
восприятии органами чувств и опыте и движимо лишь светом разума. Философия и
другие научные дисциплины относятся к сфере естественного разума, вера же,
просвещенная и ведомая Духом, видит в самом Вестнике спасения ту «полноту
благодати и истины» (см.Ин 1, 14), которую Бог пожелал явить в истории один раз
и навеки через Сына своего Иисуса Христа (см.  1 Ин 5, 9; Ин 5, 31-32).

10. Отцы Второго Ватиканского Собора, обратив свои взоры к Иисусу как к
источнику Откровения, показали спасительную природу Божественного Откровения в
истории и охарактеризовали его сущность следующими словами: «Таким образом, в
этом откровении невидимый Бог (см. Кол 1, 15; 1 Тим. 1, 17) по обилию любви
Своей обращается к людям, как к друзьям (см. Исх 33, 11; Ин 15, 14-15), и с
ними беседует (см. Вар 3, 38), чтобы пригласить их к общению с Собой и принять
их в это общение. Это домостроительство Откровения совершается действиями и
словами, внутренне связанными между собою, так что дела, совершаемые Богом в
истории спасения, являют и подтверждают учение и все, что знаменуется словами,
а слова провозглашают дела и открывают тайну, содержащуюся в них. Но внутренняя
истина, как о Боге, так и о спасении человека, сияет нам через это Откровение
во Христе, Который одновременно и Посредник, и Полнота всего Откровения»
\footnote{Догматическая конституция о Божественном Откровении Dei Verbum, 2.}.

11. Итак, Божественное Откровение вписано во времена и в анналы истории.
Воплощение Иисуса Христа произошло, когда «пришла полнота времени» (см. Гал 4,
4). Через две тысячи лет после этого события необходимо снова подтвердить, что
«в христианской вере время имеет первостепенное значение»
\footnote{Апостольское послание Tertio millennio adveniente (10 ноября 1994
г.), 10:AAS 87 (1995), 11.} , ибо именно во времени совершается все дело
творения и спасения и, прежде всего, открывается тот факт, что через воплощение
Сына Божьего уже сейчас мы можем переживать и предчувствовать то, что
завершится, когда наступит полнота времен (см. Евр 1, 2).

Истина, которую Бог поведал человеку о Самом Себе и о Своей жизни, явлена,
таким образом, во времени и в истории. Она была раз и навсегда провозглашена в
тайне Иисуса из Назарета. Это красноречиво утверждается в конституции Dei
Verbum: «Бог, многократно и многообразно говоривший в пророках, «в эти
последние дни говорил нам в Сыне» (Евр 1, 1-2). Ибо Он послал Сына Своего, то
есть Предвечное Слово, просвещающее всех людей, дабы Он обитал среди людей и
поведал им тайны Божии (см. Ин 1, 1-18). Итак, Иисус Христос, воплотившееся
Слово, посланный как «человек к людям», «говорит слова Божии» (см. Ин 3, 34) и
совершает дело спасения, которое Отец дал Ему сотворить (см. Ин 5, 36; 17, 4).
Поэтому Он, при виде Которого всякий видит Отца (см. Ин 14, 9), всем Своим
присутствием, всем, чем Он являет Себя, словами и делами, знамениями и
чудесами, особенно же смертью Своею и славным Своим воскресением из мертвых,
наконец, ниспосланием Духа истины, завершает во всей полноте Откровение»
\footnote{П. 4.}.

Следовательно, история для Народа Божьего — это путь, который он должен пройти
до конца, чтобы благодаря неустанному действию Святого Духа богооткровенная
истина явила свое содержание во всей полноте (см. Ин 16, 13). Этому снова нас
учит конституция Dei Verbum: «Церковь на протяжении веков непрерывно стремится
к полноте Божественной истины, доколе в ней Самой не завершатся слова
Божии» \footnote{П. 8.}.

12. История становится той сферой, в которой мы можем видеть деяния Бога для
людей. Он обращается к нам через то, что нам лучше всего известно и легко
постижимо, через то, что составляет саму ткань нашей повседневной жизни, без
чего мы не смогли бы понять самих себя.

Воплощение Сына Божьего дает нам возможность узреть реальность окончательного и
вечного синтеза, который человеческий разум даже не в состоянии себе
представить. Вечность вторгается во время, Тот, Кто есть Все, сосредоточивается
в маленькой частице, Бог принимает облик человека. Итак, истина, выраженная в
откровении Христа, уже не ограничена территориальными и культурными барьерами,
но открыта каждому мужчине и каждой женщине, которые хотят принять ее как
окончательное и безошибочное слово, дающее смысл их существованию. Во Христе
все люди уже имеют доступ к Отцу, ибо Христос Своей смертью и воскресением
даровал нам вечную жизнь, которую отверг первый человек, Адам (см. Рим 5,
12-15). Через это Откровение человек получает в дар совершенную истину о своей
жизни и смысле истории. «Тайна человека истинно освещается лишь в тайне
воплотившегося Слова», — утверждается в конституции Gaudium et spes
\footnote{П. 22.} . Вне этого видения вещей тайна человеческого существования
остается неразрешимой загадкой. Где же еще человеку искать ответ на те
трагические вопросы, которые он задает себе, например, о боли или о страданиях
и смерти невинных, если не в свете, который проистекает из тайны страстей,
смерти и воскресения Христа?

Разум перед лицом тайны

13. Однако нельзя забывать, что Откровение изобилует тайнами. Конечно, Иисус
всей Своей жизнью являет облик Отца, ибо Он пришел для того, чтобы поведать
тайны Божии \footnote{См. Второй Ватиканский Собор, Догматическая конституция о
Божественном Откровении Dei Verbum. 4.} ; но, несмотря на это, то знание,
которое мы имеем об этом облике, всегда характеризуется некой неполнотой и не
выходит за пределы нашего познания. Только вера позволяет нам проникнуть в
смысл тайны и сделать ее доступной для интеллекта.

Собор учит, что «Богу, дающему Откровение, нужно принести послушание веры»
\footnote{Там же, п. 5.} .  Это краткое, но содержательное изречение выражает
одну из основных истин христианской веры. Прежде всего, оно подчеркивает, что
вера — это ответ, выражающий послушание Богу. С этим связана необходимость
признания Его Божественности, трансцендентности и абсолютной свободы. Бог,
делающий Себя познаваемым, авторитетом Своей абсолютной трансцендентности,
свидетельствует о подлинности открываемых им истин. Своей верой человек дает
согласие на такое Божественное свидетельство. Это означает, что он целиком и
полностью признает истинность того, что ему было открыто, поскольку Сам Бог
является гарантом истинности. Эта истина, которую человек получает в дар и
которую сам не может требовать, вписывается в контекст неких особых
межличностных отношений и побуждает разум принять ее и признать ее глубокий
смысл. Именно поэтому акт доверия Богу Церковь всегда считала главным выбором,
который касается всей личности человека. Разум и воля максимально раскрывают
свою духовную природу, чтобы дать человеку возможность совершить действие, в
котором наиболее полно выражается его свобода \footnote{Отцы I Ватиканского
Собора учат, что послушание вере требует участия разума и воли: «Поскольку
человек всецело зависит от Бога — Творца и Господа, а сотворенный разум всецело
зависит от несотворенной истины, мы обязаны через веру выразить Богу, дающему
Откровение, наше абсолютное послушание мысли и воли» (Dei Filius, III: DS
3008).} . Следовательно, свобода не только присутствует в вере, но и является
ее непременным условием. Более того, именно вера позволяет наиболее полно
выразить свою свободу. Иначе говоря, свобода не выражается в выборе,
отвергающем Бога. Разве можно согласиться с тем, что подлинным проявлением
свободы является отказ от того, что позволяет людям в полноте выразить самих
себя? Акт веры является самым значительным выбором в жизни человека, ибо в нем
свобода приходит к достоверной истине и решает жить в соответствии с ней.

На помощь разуму, который стремится к осознанию тайны, приходят также знаки,
содержащиеся в Откровении. Они помогают более глубоко постигать истину и дают
возможность разуму самостоятельно погрузиться внутрь тайны. Во всяком случае,
эти знаки, с одной стороны, приумножают силы человеческого разума, ибо
благодаря им он может изучать тайну присущими ему средствами, которые он
ревностно оберегает, а с другой стороны, они же побуждают его выйти за их рамки
и постичь глубокий смысл, который они в себе заключают. В них, таким образом,
сокрыта истина, к которой разум должен обратиться и которую не может
игнорировать, не уничтожив самого вверенного ему знака.

Здесь мы как бы возвращаемся к аспекту Откровения, связанному с таинствами, в
особенности к евхаристическому знаку, в котором неразрывное единство между
самой вещью и ее символическим значением позволяет осознать глубину тайны.
Христос действительно присутствует и живет в Евхаристии и действует силой
Своего Духа, но таинственно, как прекрасно сформулировал это св. Фома
Аквинский: «То, что не может постичь разум, то, что не видит око, о том
свидетельствует живая вера, вопреки порядку вещей. Под различными видами, в
таинственных знаках сокрыты величайшие вещи» \footnote{Секвенция из торжества
Пресвятых Тела и Крови Христа.} . Философ Паскаль вторит ему: «Как Христос был
неизвестен среди людей, так пребывает и Его истина среди обычных суждений без
какого-либо внешнего отличия. Так пребывает и Евхаристия среди обычного хлеба»
\footnote{Penseees, 789 (ed. L. Brunschvicg).} .

Наконец, познание веры не упраздняет тайну, а только делает ее наличие более
очевидным и представляет ее как необходимый элемент жизни человека: Христос «в
самом явлении тайны Отца и Его любви в полноте являет человека самому человеку
и ясно показывает ему его наивысшее призвание» \footnote{Второй Ватиканский
Собор, Пастырская конституция о Церкви в современном мире Gaudium et spes, 22.}
, т.е. призвание к участию в жизни Триединого Бога \footnote{Второй Ватиканский
Собор, Догматическая конституция о БожественномОткровении Dei Verbum, 2.} .

14. Учение обоих Ватиканских Соборов открывает перед философией совершенно
новые горизонты. Откровение вносит в историю человечества некую отправную
точку, которой человек не может пренебрегать, если хочет постичь тайну своего
бытия; с другой стороны, это познание неизменно возвращается к тайне Бога,
которую разум не в состоянии до конца понять, а может лишь принять ее с верою.
Эти два момента определяют для человеческого разума некое пространство, в
рамках которого ему дозволяется исследовать и постигать, ибо он не
ограничивается никакой иной вещью, кроме своей конечной природы, перед лицом
бесконечной тайны Бога.

Итак, Откровение вносит в нашу историю некую универсальную и совершенную
истину, которая побуждает человеческий разум никогда не останавливаться; более
того, она подталкивает его постоянно расширять границы своего познания, пока он
не познает в совершенстве все то, что в его власти, так чтобы ничего не было
упущено. В этой решимости нас спешит поддержать один из наиболее плодовитых и
великих мыслителей в истории человечества, на которого с почтением ссылаются
как в области философии, так и богословия — св. Ансельм. В Прослогионе этот
Кентерберийский архиепископ пишет: «Когда я стал часто и упорно направлять свою
мысль к этому предмету, то подчас мне казалось, что я уже вот-вот постигну то,
что я искал, но в другой раз он вовсе ускользал от моего умственного взора, и
тогда я, вконец отчаявшись, решал прекратить свои занятия, как поиски вещи,
которую невозможно найти. Но как только я пожелал прогнать прочь эту мысль,
чтобы она не занимала мой ум понапрасну и не отвлекала от других, из которых я
мог бы извлечь больше пользы, она все больше и больше, несмотря на мое
нежелание и сопротивление, начинала назойливо навязываться мне (...). Но, увы и
мне, несчастному, одному из прочих несчастных сынов Евы, удаленных от Бога! Что
я начал делать? Что исполнил? Куда держал путь? Куда пришел? К чему стремился?
О чем воздыхаю? (...) Значит, Господи, Ты не только то, больше чего нельзя
помыслить, но Ты есть нечто большее, чем то, что можно помыслить, ибо если
можно помыслить нечто подобное и это будешь не Ты, получается, что можно
помыслить нечто большее Тебя: но этого не может быть» \footnote{Proemio и пп.
1.15: PL 158, 223-224, 226, 235.} .

15. Истина христианского Откровения, которую мы обретаем в Иисусе из Назарета,
позволяет каждому увидеть «тайну» собственной жизни. Эта наивысшая истина,
нисколько не нарушая автономию творения и его свободу, одновременно обязывает
это творение открыть себя для трансцендентного. Связь между свободой

и истиной достигает максимума, и мы полностью понимаем слова Господа: «И
познаете истину, и истина сделает вас свободными» (Ин 8, 32).

Христианское Откровение — это подлинный ориентир для человека, который вынужден
продвигаться вперед в условиях господства некоей «имманентистской»
ментальности, одновременно преодолевая препятствия, создаваемые
технократической логикой. Оно представляет собой последнюю возможность,
предоставляемую Богом, чтобы полностью вернуться к первоначальному плану любви,
началом которого было сотворение мира. Людям, желающим познать истину, если они
еще способны отвлечься от самих себя и направить свой мысленный взор выше
непосредственных житейских целей, дана возможность обрести истинно необходимое
для своей жизни на пути к истине. К вышесказанному хорошо применимы слова из
книги Второзакония: «Ибо заповедь сия, которую я заповедую тебе сегодня, не
недоступна для тебя и не далека. Она не на небе, чтобы можно было говорить:
«Кто взошел бы для нас на небо, и принес бы ее нам, и дал бы нам услышать ее, и
мы исполнили бы ее?» И не за морем она, чтобы можно было говорить: «Кто сходил
бы для нас за море, и принес бы ее нам, и дал бы нам услышать ее, и мы
исполнили бы ее?» Но весьма близко к тебе слово сие; оно в устах твоих и в
сердце твоем, чтобы исполнять его» (Втор. 30, 11-14). Эти слова нашли отклик в
известном изречении философа и богослова св. Августина: «Не выходи наружу, в
самого себя войди, ибо во внутреннем человеке обитает истина»
\footnote{Deverareligione, XXXIX, 72: CCL 32, 234.} .

Их этих предварительных рассуждений можно сделать первый вывод: истина, которую
позволяет нам познать Откровение, не есть зрелый плод или итог какого-либо
чисто человеческого размышления. Напротив, эта истина есть бескорыстный дар,
она сама будит мысль и требует, чтобы ее приняли как проявление любви. Эта
богооткровенная истина вписана в нашу историю как предвестие окончательного и
совершенного видения, которое Бог собирается дать тем, кто в Него верит и ищет
Его с искренним сердцем. Итак, конечная цель жизни каждого человека — это
предмет изучения как философии, так и богословия. Обе эти дисциплины, несмотря
на различия в методах и содержании, указывают «путь жизни»

(Пс 16 (15), 11), в конце которого, как учит вера, нас ожидает полная и
непреходящая радость созерцания Триединого Бога.

\section{Верую, чтобы понимать}

«Премудрость все знает и все разумеет» (Прем 9, 11) 16. Уже в Священном Писании
встречаются удивительно точные утверждения, которые показывают, насколько
глубоко познание веры связано с рациональным познанием. Это, прежде всего,
подтверждается на примере Книг Премудрости. Непредвзятый читатель Писания
явственно видит, что эти книги хранят не только веру Израиля, но и духовные
богатства ныне уже не существующих культур. Кажется, что в силу особого замысла
мы снова слышим голос Египта и Месопотамии, и некоторые элементы цивилизаций
древнего Востока оживают на этих страницах, содержащих множество необычайно
глубоких мыслей.

Не случайно, что, когда автор Книг Премудрости желает дать образ мудреца, он
представляет его как человека, любящего и ищущего истину: «Блажен человек,
который упражняется в мудрости и в разуме своем поучается святому. Кто
размышляет в сердце своем о путях ее, тот получит разумение и в тайнах ее.
Выходи за нею, как ловчий, и строй засаду на путях ее. Кто приклоняется к окнам
ее, тот послушает и при дверях ее. Кто обращается вблизи дома ее, тот вобьет
гвоздь и в стенах ее; поставит палатку свою подле нее, и будет обитать в жилище
благ. Он положит детей своих под кровом ее и будет иметь ночлег под сенью ее.
Он прикроется ею от зноя и будет жить в славе ее» (Сир 14, 21-27).

Как видим, по мнению богодухновенного автора, стремление к познанию есть
свойство, присущее всем людям. Благодаря наличию интеллекта всем — как
верующим, так и неверующим — дана способность черпать из «глубоких вод»
познания (см. Притч 20, 5).

Несомненно, в древнем Израиле мир со всеми его явлениями познавался без
абстрагирования от восприятия вещей, в отличие от познавательных методов,
свойственных ионийским философам и египетским мудрецам. Правоверный иудей тем
более не использовал методы познания, свойственные современной эпохе,
стремящейся разделить знания на отдельные дисциплины. Несмотря на это,
библейский мир внес свой вклад в обширную сокровищницу человеческого познания.

В чем он заключается? Отличительной особенностью, характеризующей библейский
текст, является убеждение, что между рациональным познанием и познанием веры
существует глубокая и неразрывная связь. Мир и то, что в нем происходит, а
также история и судьбы народа — это действительность, которую следует
постигать, анализировать и оценивать с помощью средств, имеющихся в
распоряжении разума, но так, чтобы вера не была отстранена от этого процесса.
Вера необходима не для того, чтобы лишить разум автономии или ограничить его
поле деятельности, но лишь для того, чтобы объяснить человеку, что в этих
судьбах присутствует и действует Бог Израиля. Итак, подлинное познание мира и
исторических событий невозможно, если отсутствует вера в Бога, Который
действует в них.

Вера обостряет внутреннее зрение и просвещает разум, позволяя ему заметить в
веренице событий присутствие Провидения. Следующие слова из книги Притчей могут
служить тому прекрасной иллюстрацией: «Сердце человека обдумывает свой путь, но
Господь управляет шествием его» (Притч 16, 9). Это значит, что человек,
просвещенный светом разума, может найти свой путь, легко пройти его, избегая
препятствий, и достичь цели, если правым сердцем упорядочит свои искания в
области веры. Следовательно, нельзя разделить веру и разум, не лишив человека
возможности должным образом познать самого себя, мир и Бога.

17. Поэтому нет оснований для соперничества веры и разума: они взаимно
дополняют друг друга и каждое имеет свою область, в которой осуществляется. На
это нам вновь указывает книга Притчей, автор которой восклицает: «Слава Божия —
облекать тайною дело, а слава царей — исследовать дело» (Притч 25, 2). В рамках

единственной в своем роде взаимосвязи Бог и человек все же пребывают в своем
собственном мире. В Боге заключено начало всего сущего, в Нем — полнота тайны,
и это является Его славой; задача человека — исследовать своим разумом истину,
и в этом, безусловно, заключается его достоинство. Последнюю деталь в эту
картину привносит псалмопевец, восклицая: «Как возвышенны для меня помышления
Твои, Боже, и как велико число их! Стану ли исчислять их, но они многочисленнее
песка; когда я пробуждаюсь, я все еще с Тобою» (Пс 139 (138), 17-18). Желание
познания настолько сильно и пробуждает такую энергию, что дух человека, хотя и
ощущает предел, который он самостоятельно не в силах преодолеть, тоскует по
бесконечному благу, скрытому за ним, ибо чувствует, что в нем заключен
исчерпывающий ответ на любой еще не решенный вопрос.

18. Можно сказать, что размышления Израиля открыли разуму путь к исследованию
тайны. В Божественном Откровении человек мог постигать глубины, в которые
безуспешно пытался проникнуть разум. Усовершенствуясь в этом более глубоком
способе познания, избранный народ понял, что разум должен придерживаться
некоторых основных принципов, чтобы наиболее полно выражать свою природу.
Первый принцип состоит в том, чтобы разум признал следующую истину: человеку
предстоит совершить путь, которого он не может избежать; второй принцип основан
на свидетельстве совести и гласит, что на этот путь нельзя вступать с гордыней,
т.е. полагая, что цель может быть достигнута собственными силами; третий
принцип основан на «страхе Божием», который принуждает разум признать высшую
трансцендентность Бога и одновременно Его заботливую любовь в управлении миром.

Если человек отходит от этих принципов, он может потерпеть неудачу и в
результате оказаться «в положении глупца». Согласно Библии, такая глупость
угрожает жизни. Глупец уверен, что обладает обширными познаниями, а в
действительности не может направить свой ум на вещи необходимые. Поэтому он не
может должным образом упорядочить свой разум (см. Притч 1, 7) и принять
правильную установку по отношению к себе и окружающим вещам. Когда он, наконец,
начинает утверждать, что «нет Бога» (см. Пс 14 (13), 1), он ясно дает понять,
насколько ничтожны его позна-

ния и как он далек от полной истины о вещах, их происхождении и предназначении.

19. Некоторые важные тексты, которые проливают свет на этотвопрос, находятся в
Книге Премудрости Соломона. Богодухновенный автор говорит в них о Боге,
которого можно постичь черезприроду вещей. Для древних изучение природы в
высшей степенисозвучно с философским познанием. Указав вначале, что
человексилой своего разума может «познать устройство мира и действиестихий,
(...) круги годов и положение звезд, природу животныхи свойства зверей» (Прем
7, 17, 19-20), одним словом, может философствовать, священный автор делает еще
один важный шаг вперед: повторяя тезис греческой философии, к которой он,
по-видимому, апеллирует в этом фрагменте, автор утверждает, что именночерез
рациональное познание природы можно достичь Творца:«...ибо от величия красоты
созданий сравнительно познается Виновник бытия их» (Прем 13, 5). Здесь отмечена
первая ступень Божественного Откровения, то есть замечательная «книга
природы»:читая ее с помощью средств, которыми располагает разум, человекможет
достичь познания Творца. Если человек с помощью своегоинтеллекта не может
распознать в Боге Творца всего сущего, то причина кроется не столько в
отсутствии необходимых средств, сколько в препятствиях, созданных его свободной
волей и грехами.

20. С этой точки зрения следует по достоинству оценить разум,но не
переоценивать его. Знания, полученные им, могут быть истинными, но они обретают
полное значение только тогда, когдасвязаны с верой: «От Господа направляются
шаги человека; человеку же как узнать путь свой?» (Притч 20, 24). Поэтому,
согласно Ветхому Завету, вера освобождает разум, ибо позволяет ему в
соответствии с его принципами найти объект познания и включить егов высший
порядок вещей, в котором все обретает смысл. Однимсловом, с помощью разума
человек находит истину, поскольку,просвещенный верой, открывает глубокий смысл
всего сущего,в особенности своего существования. Поэтому богодухновенныйавтор
справедливо видит источник истинного познания в страхеБожием: «Начало мудрости
— страх Господень» (Притч 1, 7; см.Сир 1, 14).

«Приобретай мудрость, приобретай разум» (Притч 4, 5)

21. В традиции Ветхого Завета познание основано не только навнимательном
изучении человека, мира и истории, но и предусматривает существование
необходимой связи с верой и Откровением.В этом заключается вызов, который
должен был принять избранный народ и на который он должен был дать ответ.
Размышляяо своем положении, ветхозаветный человек открыл, что может понять
самого себя лишь как пребывающего в связи — с самим собой,с народом, с миром и
с Богом. Эта открытость в отношении тайны,приобретенная им через Откровение,
стала для него источникомистинного познания, которое позволило его разуму
проникнутьв область бесконечного и благодаря этому получить новые неожиданные
познавательные возможности.

В этих поисках богодухновенный автор встречал немало препятствий, связанных с
ограниченностью человеческого разума. Это заметно в Книге Притчей в отрывке об
усталости человека, который пытается постичь тайные замыслы Бога (см. Притч 30,
1-6). Однако, несмотря на усталость, верующий не сдается. Он черпает силы для
стремления к истине из убеждения, что Бог создал его, «чтобы он исследовал»
(см. Екк 1, 13), и его миссия заключается в том, чтобы не оставлять ничего не
исследованным, невзирая на постоянную угрозу сомнений. Находя помощь у Бога, он
всегда и везде стремится к прекрасному, доброму и истинному.

22. В первой главе Послания к Римлянам св. апостол Павел помогает нам понять,
насколько глубока та решимость, о которой говорится в Книгах Премудрости.
Пользуясь разговорным языком,апостол делает философское умозаключение, в
котором выраженаглубокая истина: наблюдение сотворенных вещей «очами
разума»может привести к познанию Бога, ибо Он Сам через Свои творенияпозволяет
разуму узреть Его «силу» и «Божество» (см. Рим 1, 20).Этим выражается признание
того, что человеческий разум обладает способностью, которая, по всей
вероятности, преодолевает егоестественную ограниченность: она не только не
замкнута в сферечувственного восприятия, ибо может делать его предметом
критической оценки, но и, анализируя чувственные представления, может открыть
причину, которая дает начало всем видимым вещам.

На языке философии можно сказать, что в этом важном тексте Павла постулируется
способность человека к метафизическому мышлению.

По мнению апостола, по первоначальному замыслу Творца разум должен был обладать
способностью свободно преодолевать границы чувственного восприятия и находить
глубочайший источник всего сущего, то есть Творца. Но по причине непослушания,
которое проявилось в том, что человек предпочел получить полную и абсолютную
независимость от Того, Кто его создал, он потерял способность обращаться к Богу
Творцу.

В книге Бытия образно описывается первоначальное состояние человека, когда
повествуется о том, что Бог поселил его саду Едемском, посреди которого росло
дерево «познания добра и зла» (см. Быт 2, 17). Значение этого символа понятно:
человек не мог самостоятельно решать, что есть добро, а что — зло, и должен был
следовать некоему высшему принципу. Слепая гордыня наших прародителей создала у
них ложное убеждение, что они независимы и самостоятельны, а, следовательно,
могут обойтись без познания, исходящего от Бога. Последствия их непослушания
коснулись всех людей и причинили человеческому разуму раны, которые с тех пор
стали препятствием на его пути к истине. Способность к познанию была ослаблена
из-за того, что человек отвернулся от Того, Кто является источником и началом
истины. Об этом также говорит апостол, показывая, насколько человек «осуетился
в умствованиях своих» из-за греха, а его мышление стало извращенным и привело
ко лжи (см. Рим 1, 21-22). Очи разума уже не могли ясно видеть: постепенно
разум поработил самого себя. Пришествие Христа стало спасительным событием,
поскольку избавило разум от слабости, разорвав цепи, которыми он сам себя
сковал.

23. Поэтому отношение христианина к философии требует внимательного
рассмотрения. В Новом Завете, особенно в посланиях св. апостола Павла, ясно
говорится о том, что «мудрость мира сего» противоположна мудрости Божией,
явленной в Иисусе Христе. Глубина богооткровенной мудрости разрывает тесный
круг привычного мышления, ибо оно не в силах ее адекватно выразить.

Эта дилемма со всей остротой представлена в начале Первого

послания к Коринфянам. Распятие Сына Божьего является историческим событием,
перед которым являют свое бессилие все попытки разума сформулировать смысл
бытия в чисто человеческих категориях. Ключевая проблема, которая является
вызовом для любой философии, — это смерть Иисуса Христа на кресте, ибо любая
попытка объяснить с помощью чисто человеческой логики спасительный план Отца
обречена на неудачу. «Где мудрец? Где книжник? Где совопросник века сего? Не
обратил ли Бог мудрость мира сего в безумие?» (1 Кор 1, 20) — настойчиво
спрашивает апостол. Чтобы могло свершиться то, чего желает Бог, уже
недостаточно знаний мудрецов, необходимо быть готовым к принятию совершенно
новой реальности: «Но Бог избрал немудрое мира, чтобы посрамить мудрых; (...) и
незнатное мира,и уничиженное, и ничего не значащее избрал Бог, чтобы упразднить
значащее» (1 Кор 1, 27-28). Человеческая мудрость не хочет признать, что ее
слабость является условием ее силы, но св. Павел не боится сказать: «Когда я
немощен, тогда силен». Человек не может понять, каким образом смерть может быть
источником жизни и любви, но Бог выбрал именно то, что разум считает «безумием»
и «соблазном», чтобы явить тайну Своего спасительного замысла. Пользуясь языком
современных ему философов, Павел достигает вершины своего учения, когда
формулирует его в виде провозглашаемого им парадокса: «Бог избрал
несуществующее, чтобы упразднить существующее» (1 Кор 1, 28). Чтобы выразить
бескорыстие любви, явленной Христом на кресте, апостол дерзает употребить
наиболее радикальные выражения, которые использовали философы, размышляя о
Боге. Разум не может лишить смысла тайну любви, которую являет собой крест, ибо
лишь крест может дать человеческому разуму окончательный ответ, который он
ищет. Святой апостол Павел называет правилом истины, а тем самым спасения, не
мудрость слов, но само Слово мудрости.

Итак, мудрость креста не ограничена культурными границами, которые ей пытаются
навязать извне, но повелевает всякому принять скрытую в ней универсальную
истину бытия. Какой вызов брошен нашему разуму и какую пользу он может
получить, если его примет! Философия, которая и своими собственными силами
может заметить, что человек постоянно превосходит самого себя

в стремлении к истине, с помощью веры может решиться на принятие в «безумии»
креста критического суждения о тех, которые ложно считают себя обладающими
истиной, хотя на самом деле укладывают ее в прокрустово ложе своих систем.
Необходимость проповеди Христа распятого и воскресшего воздвигает между верой и
разумом некий утес, который может стать причиной кораблекрушения, но за которым
раскрывается бесконечный океан истины. Здесь четко определены границы веры и
разума, но также определено то место, где они могут встретиться.

\section{Понимаю, чтобы верить}

В поисках истины

24. В Деяниях Апостолов евангелист Лука повествует, как однажды во время одного
из миссионерских путешествий апостол Павел прибыл в Афины. Этот город,
являвшийся столицей философов, изобиловал статуями различных богов. Павел
обратил внимание на некий алтарь и сразу нашел исходный пункт для того общего
основания, с которого он начал свою проповедь: «Афиняне, — сказал он, — по
всему вижу я, что вы как бы особенно набожны. Ибо, проходя и осматривая ваши
святыни, я нашел и жертвенник, на котором написано: «неведомому Богу». Сего-то,
Которого вы, не зная, чтите, я проповедую вам» (Деян 17, 22-23).

После этого вступления Павел говорит о Боге как о Творце, как о Том, Кто все
превосходит и все животворит. Далее он утверждает: «От одной крови Он произвел
весь род человеческий для обитания по всему лицу земли, назначив
предопределенные времена и пределы их обитанию, дабы они искали Бога, не ощутят
ли Его и не найдут ли, хотя Он и недалеко от каждого из нас» (Деян 17, 26-27).

Апостол демонстрирует истину, которой Церковь всегда прида-

вала большое значение: в глубине сердца человека заключено стремление к Богу,
тоска по Нему. Об этом напоминает богослужение в Великую Пятницу, когда в
ектении за неверующих мы молимся: «Всемогущий, вечный Боже, Ты создал всех
людей, чтобы они всегда с жадностью искали Тебя, а, найдя, обретали покой»
\footnote{«Ut te semper desiderando quaererent et inveniendo quiescerent»:
MissaleRomanum.} .  Итак, существует путь, который человек, если хочет, может
преодолеть; в его начале лежит способность разума оторваться от условных вещей
и устремиться к бесконечности.

В различные эпохи человек по-разному выражал это свое глубинное желание.
Литература, музыка, живопись, скульптура, архитектура и другие плоды его
творческого интеллекта стали средствами выражения неудовлетворенности,
побуждающей человека к неустанному поиску. Это стремление нашло выход в
философии, которая с помощью характерных для нее научных методов и средств
выразила это универсальное желание человека.

25. «Все люди желают знания» \footnote{Аристотель, Метафизика, I, 1.}, а
объектом этого желания является истина. Даже в повседневной жизни мы замечаем,
как сильно каждый человек стремится познать истинное положение вещей, не
ограничиваясь информацией из вторых рук. Человек — это единственное существо во
всем видимом, созданном мире, которое не только способно знать, но осознает,
что знает, и поэтому стремится познать истину тех вещей, которые составляют
предмет его восприятия. Никто не может равнодушно относиться к тому, истинны
его знания или нет. Если человек выясняет, что они ложны, он отвергает их; если
убеждается в их правдивости, испытывает удовлетворение. Именно об этом говорит
св. Августин: «Многих знаю я, кто охотно обманывает, и никого, кто хотел бы
обмануться» \footnote{Confessiones, X, 23, 33: CCL 27, 173.} . Справедливо
считается, что человек достиг зрелого возраста, если самостоятельно может
отличить правду от лжи, вынося таким образом собственное суждение об истинном
положении вещей.  Именно это мотивирует разнообразные поиски, особенно в
области естественных наук, которые в последние столетия принесли столь
значительные результаты и способствовали подлинному прогрессу человечества.

В практической сфере ведутся не менее важные поиски, чем в теоретической: я
имею в виду поиски истины о добре, которое

надлежит осуществить. Ибо, совершая этические поступки и действуя в
соответствии со своей свободной и правой волей, человек вступает на путь,
ведущий к счастью, и стремится к совершенству. И в этом случае речь идет об
истине. Это убеждение я выразил в энциклике Veritatis splendor: «Не существует
нравственности без свободы. (...) Если существует право искать истину
собственным путем, тем более существует для каждого серьезная нравственная
обязанность искать истину и придерживаться ее, когда мы ее обрели» \footnote{П.
34: AAS 85 (1993), 1161.} .

Следовательно, необходимо, чтобы ценности, которые выбирает человек и к которым
стремится, были истинными, ибо только благодаря истинным ценностям он может
совершенствоваться, всецело развивая свои природные способности. Человек не
найдет истинных ценностей, если замкнется в себе, он должен раскрыться и искать
их также в сферах, трансцендентных человеческой природе. Это обязательное
условие должен выполнить каждый, чтобы стать самим собой и развиваться как
взрослая и зрелая личность.

26. Сначала истина открывается человеку в виде вопроса: Имеет ли жизнь смысл? К
чему она устремлена? На первый взгляд, существование личности кажется
совершенно лишенным смысла. Нет даже необходимости обращаться к философии
абсурда или к провокационным вопросам из Книги Иова, чтобы усомниться в наличии
смысла жизни. Наши собственные или чужие страдания, многочисленные события,
которые разум не в состоянии объяснить, — этого достаточно, чтобы неизбежно
возник драматический вопрос о смысле жизни \footnote{См. Иоанн Павел II,
Апостольское послание Salvifici doloris (11 февраля1984 г.), 9: AAS 76 (1984),
209-210.}  . К этому необходимо добавить, что первой абсолютно достоверной
истиной нашего бытия, помимо того, что мы уже существуем, является неизбежность
смерти. Эта тревожная действительность побуждает нас искать исчерпывающий
ответ. Каждый желает — и должен — узнать правду о том, что его ждет. Мы хотим
знать, прекращает ли смерть полностью наше существование или в нас есть нечто,
что не исчезает со смертью; можно ли надеяться на жизнь после смерти или нет.
Симптоматично, что после смерти Сократа философская мысль получила особый
импульс, который определил ее направление на более чем две тысячи лет. Не
случайно философы, размышляя о факте смерти, по-

стоянно ставили перед собой вышеуказанный вопрос, равно как и вопросы о смысле
жизни и бессмертии.

27. Никто не может избежать этих вопросов — ни философ, ниобычный человек. От
ответа на них зависит решающий этап поисков: возможно ли найти универсальную и
абсолютную истину. Посвоей природе любая истина, даже частичная, если она
аутентична,представляется универсальной и абсолютной. То, что является истиной,
должно быть истиной всегда и для всех. Однако кроме универсальности такого рода
человек ищет абсолют, который смог быдать ответ на его вопросы и придать смысл
всем его поискам: некоевысшее сущее, являющееся основанием всякой вещи. Другими
словами, он ищет окончательный ответ, некую высшую ценность, после которой уже
нет и не может быть дальнейших вопросов и каких-либо добавлений. Мнения могут
нравиться, но они не приносят удовлетворения. В жизни каждого человека
наступает момент,когда — признает он это или нет — он чувствует необходимость
утвердить свое существование на абсолютной истине, которая порождает
уверенность и более не подлежит сомнению.

На протяжении веков философы старались открыть и выразить подобную истину,
создавая различные системы и школы. Однако кроме философских учений существуют
другие формы, в которых человек пытается выразить какую-то свою «философию»:
это его личные убеждения и опыт, семейные и культурные традиции или какой-то
особый экзистенциальный путь, на котором он доверят свою судьбу авторитету
духовного учителя. В любой из этих форм видно желание найти достоверную истину,
имеющую абсолютную ценность.

Различные аспекты истины о человеке

28. Но не всегда, и это следует признать, поиски истины отличаются ясностью и
последовательностью. Естественная ограниченность разума и непостоянство духа
часто осложняют поиски и направляют их по ложному пути. Другие стремления могут
оказаться сильнее истины. Бывает, что человек даже избегает истины,
едваначинает познавать ее, ибо боится ее требований. Но даже когда он

избегает ее, истина влияет на его жизнь, ведь человек не может строить свою
жизнь на неопределенности, сомнении или лжи, поскольку такая жизнь постоянно
рождает страх и беспокойство. Следовательно, человека можно определить как
того, кто ищет истину.

29. Невозможно себе представить, чтобы стремление к поискам,так глубоко
укоренившееся в натуре человека, оказалось бесцельным и лишенным смысла. Уже
сама способность искать истину изадавать вопросы представляет собой первый
ответ. Человек не начинает искать то, о чем ничего не знает и что считает
совершеннонедоступным. Только надежда получить какой-либо ответ убеждает его
сделать первый шаг. Действительно, именно так происходитв научных
исследованиях. Если ученый, руководствуясь интуицией, начинает искать
логическое и поддающееся проверке объяснение определенного явления, он с самого
начала питает твердую надежду, что найдет ответ; поэтому его не останавливают
неудачи. Онне считает свою интуицию бесполезной только потому, что не удалось
достичь цели; он правильно полагает, что пока ему не удалосьнайти правильное
решение.

То же самое нужно сказать и о поисках истины в сфере вечных вопросов. Жажда
истины настолько глубоко укоренена в сердце человека, что отказ удовлетворить
ее привел бы к экзистенциальному кризису. Достаточно понаблюдать за
повседневной жизнью, чтобы заметить, что каждый из нас задается извечными
вопросами и одновременно хранит в сердце по крайней мере предвосхищение ответов
на них. Речь идет об ответах, истинность которых мы сознаем, ибо ясно, что они
существенно не отличаются от ответов, к которым пришли многие другие.
Разумеется, не все постигнутые истины одинаково ценны. Однако совокупность
достигнутых результатов подтверждает, что человек, вообще говоря, способен
прийти к истине.

30. Теперь следует кратко рассмотреть различные виды истин.Большинство из них
составляют истины, основанные на непосредственной очевидности или
подтверждаемые экспериментальнымпутем; эти истины касаются повседневной жизни и
научных исследований. К иной области относятся истины философского характера,
которые человек постигает благодаря способности разума к умо-зрительным
рассуждениям. Существуют, наконец, религиозные истины, основания которых в
какой-то мере обретаются в философии. Они заключены в ответах на извечные
вопросы, которые даются в различных религиозных традициях \footnote{См. Второй
Ватиканский Собор, Декларация об отношении Церкви к нехристианским религиям
Nostra aetate, 2.}  .

Что касается философских истин, следует пояснить, что они содержатся не только
в доктринах (часто недолговечных) профессиональных философов. Как было уже
сказано, каждый человек — в какой-то степени философ, и у него есть собственные
философские понятия, которыми он руководствуется в жизни: каждый по-своему
формирует свое мировоззрение, решает вопрос о смысле собственной жизни и в этом
свете интерпретирует свой личный опыт и направляет свои поступки. Именно сейчас
следует поставить вопрос о связи философско-религиозных истин с истиной,
явленной в Иисусе Христе. Прежде чем мы ответим на этот вопрос, необходимо
продолжить дальнейший анализ философского познания еще в одном аспекте.

31. Человек не создан для того, чтобы жить в одиночестве. Он рождается и растет
в семье, дабы с возрастом благодаря собственным стараниям включиться в
общественную жизнь. С момента рождения он связан с различными традициями, от
которых перенимает не только язык и культурный багаж, но также и многочисленные
истины, в которые верит почти инстинктивно. Тем не менее, в период отрочества и
созревания личности человек может усомниться в этих истинах и подвергнуть их
переосмыслению с помощью особой критической деятельности разума. Однако ничто
не препятствует тому, чтобы после завершения этого процесса человек вновь
принял эти же истины, основываясь на собственном опыте или рассуждениях.
Несмотря на это, в жизни человека намного больше истин, в которые он просто
верит, чем тех, достоверность которых он установил самостоятельно. Кто может
проверить бесчисленные результаты научных исследований, которые лежат в основе
современной жизни? Кто может самостоятельно проконтролировать поток информации,
которая ежедневно поступает со всех концов света и обычно считается правдивой?
Кто, наконец, может повторить опыты и размышления, в результате которых
человечество накопило сокровища мудрости и религиозно-

го чувства? Человек — существо ищущее — есть также тот, кто живет верой.

32. Веруя, каждый человек доверяет знаниям, полученным другими. Здесь следует
признать наличие некоторого противоречия: с одной стороны, кажется, что знание,
основанное на вере, является несовершенным и должно постепенно
усовершенствоваться за счет приобретаемых свидетельств; с другой — вера часто
богаче по сравнению с обычным познанием, основанным на бесспорных
доказательствах, ибо она обогащена межличностными отношениями и использует не
только познавательные способности интеллекта, но и более глубокую способность
доверять другим людям, устанавливать с ними более тесные и продолжительные
отношения.

Необходимо подчеркнуть, что в этих межличностных отношениях главным предметом
поиска являются не эмпирические или философские истины. В основном при этом
ищут истину о самой личности — кем она является и как она внутренне
раскрывается, — ибо совершенство человека состоит не только в получении
абстрактных знаний об истине, но и в поддержании отношений с другим человеком,
которые выражаются в самоотдаче и верности другому. В этой верности, которая
позволяет человеку пожертвовать самим собой, человек находит уверенность и
твердость духа. Но одновременно познание, опирающееся на доверие, в основе
которого лежит уважение друг к другу, связано с истиной: веруя, человек
доверяет истине, которую предлагает другая личность.

Сколько примеров можно было бы привести, чтобы это наглядно представить! Но мне
сразу вспоминается свидетельство мучеников, ибо мученик является наиболее
подлинным свидетелем истины о существовании. Он знает, что благодаря встрече с
Иисусом Христом он открыл истину о собственной жизни, и никто и ничто не сможет
лишить его этой уверенности. Ни страдания, ни жестокая смерть не заставят его
отречься от истины, которую он открыл при встрече с Христом. Вот почему
свидетельство мучеников до сих пор не перестает восхищать, находит признание,
привлекает внимание и побуждает к подражанию. Вот почему мы верим их словам: мы
видим в них свидетельство любви, которая не нуждается в длинных рассуждениях,
чтобы убедить, поскольку дает каждому из нас то, что в глубине души мы уже
считаем истиной и давно ищем. Мученик вызывает у нас доверие, ибо говорит о
том, что мы видим, и делает явным то, что мы хотели бы выразить с той же силой.

33. Итак, мы видим, что обсуждаемая проблема постепенно вырисовывается во всех
аспектах. В силу своей природы человек ищетистину. Целью этих поисков является
не только познание частичных истин, которые зависят от эмпирических событий и
научныхфактов; человек ищет истинное благо не только применительно ксвоей
сиюминутной выгоде. Он ищет более глубокую истину, которая может открыть ему
смысл жизни; следовательно, речь идет опоисках, которые могут достигнуть цели
лишь в абсолюте \footnote{Это мышление я изучаю давно и говорил о нем по разным
    поводам. «Чтоесть человек и что польза его? Что благо его и что зло его?»
    (Сир. 18, 7). (...)Такие вопросы звучат в душе каждого человека. Это лучше
    всего доказываеттот факт, что поэтический гений любой эпохи и любого
    народа, являющийся своего рода пророком человечества, неустанно задает
    «серьезные вопросы», которые делают человека воистину человеком. Они
    выражают неотложную необходимость найти смысл жизни, в каждый момент, на
    важных и решающих этапах, а также в самое обычное время. Эти вопросы
    свидетельствуют о глубокой разумности существования человека, поскольку
    побуждают разум и волю человека свободно искать решение, которое могло бы
    придать смысл его жизни. Поэтому они являются самым возвышенным проявлением
    природы человека, и, что за этим следует, ответ на них является критерием
    глубины его вовлеченности в собственную жизнь. Особенно тогда, когда
    «причина вещей» полностью исследуется в процессе поиска окончательного и
    исчерпывающего ответа, разум человека достигает высшей точки и готов
    принять религию, ибо религиозность — это самое возвышенное проявление
    человеческой личности, поскольку является высшим проявлением его разумной
    природы. Она проистекает из стремления человека к истине и представляет
    собой основу свободного и личного поиска Божественного» (Общая аудиенция 19
октября 1983 г., 1-2: Insegnamenti, VI, 2 (1983), 814-815).}  . Благодаря
способностям ума человек может найти и распознать такуюистину. Поскольку эта
истина имеет существеннейшее значениедля его жизни, она достигается не только в
результате размышлений, но и через доверие другим людям, которые могут
ручатьсяза ее надежность и подлинность.  Способность и решимость доверить
самого себя и свою судьбу другому человеку является, несомненно, одним из
наиболее важных и выразительных действийс точки зрения антропологии.

Не следует забывать, что в своих поисках разум должен основываться на
доверительном диалоге и искренней дружбе. Атмосфера подозрительности и
недоверия, которая иногда сопутствует теоретическим исследованиям, противоречит
учению древних философов, по мнению которых дружеские отношения более всего
способствуют здравому философствованию.

Из сказанного ранее следует, что человек занимается поиском, который не может
завершить собственными силами: он ищет истину и того, кому можно довериться.
Христианская вера приходит ему на помощь и указывает конкретный путь к
достижению цели этих поисков. Она помогает ему преодолеть стадию наивной веры и
приводит его к состоянию благодати, в котором он может приобщиться к тайне
Христа, а в Нем истинно и полно познать Триединого Бога. Итак, в Иисусе Христе,
Который есть Сама Истина, вера находит главное призвание человечества,
позволяющее осуществить то, чего оно желает и о чем тоскует.

34. Истина, которую Бог явил нам в Иисусе Христе, не противоречит истинам,
которые можно постичь в результате философских размышлений. Наоборот, эти два
способа познания ведут к полноте истины. Единство истины является основным
постулатом человеческого разума, выраженным в законе противоречия. Откровение
убеждает нас в этом единстве, указывая, что Бог Творец — это также Бог истории
спасения. Один и тот же Бог, являющийся основой и залогом познаваемости и
рациональности естественного порядка вещей, на который с доверием опираются
ученые \footnote{«[Галилей] ясно утверждал, что две истины, т.е. вера и наука,
    никогда немогут противоречить друг другу. «Священное Писание и природа
    происходят от Слова Божьего. Первое продиктовано Святым Духом, а вторая
    является верной исполнительницей велений Бога», — так он писал в письме
    отцуБенедетто Кастелли 21 декабря 1613 г. Отцы Второго Ватиканского
    Собораучат тому же и используют подобные выражения: «... методическое
    исследование во всех дисциплинах, если оно производится (...) согласно
    нравственным нормам, по существу не может никогда противоречить вере,ибо
    для земных ценностей и ценностей веры один источник — Бог»(Gaudium et spes,
    36). В своих научных исследованиях Галилей ощущал присутствие Бога, Который
    его вдохновлял, предупреждал и помогал ему в егодогадках, действуя в
    глубине его разума» (Иоанн Павел II, Выступление в Папской Академии Наук 10
ноября 1979 г.: Insegnamenti, II, 2 (1979), 1111-1112).}  , явлен и как Отец
Господа нашего Иисуса Христа.  Это единство естественной и явленной истины
находит живое и личное воплощение во Христе по слову апостола: «Истина, которая
в Иисусе...» (Еф 4, 21; см. Кол 1, 15-20). Он — Предвечное Слово, Которым все
сотворено, и одновременно Воплощенное Слово, Которое в своей целостной Личности
\footnote{См. Второй Ватиканский Собор, Догматическая конституция о
Божественном Откровении Dei Verbum, 4.}  являет Отца (см. Ин 1, 14. 18). То,
что человеческий разум ищет «не зная» (Деян 17, 23), можно найти только через
Христа, ибо то, что в Нем раскрывается, является «полнотой истины» (см. Ин 1,
14-16) каждого творения, которое в Нем и Им было сотворено, а следовательно, и
в Нем «стоит» (см. Кол 1,17). 35. В контексте этих рассуждений следует вплотную
заняться изучением соотношения между богооткровенной истиной и философией. Это
отношение следует рассматривать с двух точек зрения, поскольку истина, явленная
нам в Откровении, одновременно является истиной, которую необходимо осознать в
свете разума.  Только в свете такого двойного толкования можно определить
подлинную связь между богооткровенной истиной и философскими знаниями. Поэтому
рассмотрим сначала соотношение веры и разума на протяжении столетий. На этом
основании можно будет установить принципы, являющиеся отправными точками,
которые помогут определить подлинное отношение между этими двумя видами
познания.

\section{О соотношении между верой и разумом}

Важнейшие этапы встречи веры и разума

36. Как свидетельствуют Деяния Апостолов, с самого начала христианскому учению
приходилось сталкиваться с различными течениями философской мысли того времени.
Эта книга описывает дискуссию, которую св. Павел вел в Афинах с некоторыми
эпикурейскими и стоическими философами (Деян 17, 18). Экзегеза его проповеди в
Ареопаге показывает многочисленные аллюзии на распространенные в то время
суждения, относящиеся в основном к стоической философии. Разумеется, это не
случайно. Если первые христиане хотели, чтобы язычники их поняли, они не могли
ссылаться лишь на «Моисея и пророков»; они также должны были использовать
естественное познание Бога и глас совести каждого человека (см. Рим 1, 19-21;
2, 14-15; Деян 14, 16-17). Но поскольку в языческой религии это естественное
познание было затемнено идолопоклонством (см. Рим 1, 21-32), апостол посчитал
более целесообразным сослаться в своей проповеди на мысли философов, которые с
самого начала противопоставляли мифам и мистическим культам концепции,
подчеркивающие Божественную трансцендентность.

Классические философы стремились, прежде всего, очистить представления человека
о Боге от мифологических наслоений. Как известно, греческая религия, как и
большинство космических религий, носила политеистический характер и приписывала
Божественные свойства даже предметам и явлениям природы. Попытки человека
постичь происхождение божеств, а тем самым вселенной, первоначально выражались
в поэзии. Древние теогонии остаются первым свидетельством таких поисков.
Задачей же первых создателей философии было показать связь разума и религии.
Обращая свой взор ввысь, к универсальным принципам, они уже не останавливались
на античных мифах, а хотели рационально обосновать свою веру в божество. Таким
образом, они встали на путь, который, оставив в стороне прежние частные
традиции, характеризо-

вался неким прогрессом, отвечающим требованиям всеобщего разума. Целью этого
прогресса было критическое суждение о предметах веры. Это начинание благотворно
повлияло на саму первоначальную концепцию божества. Предрассудки были
изобличены, и религия, по крайней мере, частично, была очищена с помощью
рационального анализа. Именно на этой основе Отцы Церкви начали плодотворный
диалог с древними философами, открыв путь для проповедования и познания Бога,
явленного в Иисусе Христе.

37. Говоря об этом сближении христиан с философией, следуетпомнить об их
настороженности по отношению к иным элементамязыческой культуры, например к
гностицизму. Философию какпрактическую мудрость и школу жизни было легко
отождествитьс некоторыми видами таинственных, эзотерических знаний,
предназначенных для немногочисленных избранных. Несомненно,именно такие
эзотерические спекуляции имел в виду св. Павел,когда предостерегал колоссян:
«Смотрите, братия, чтобы кто не увлек вас философиею и пустым обольщением, по
преданию человеческому, по стихиям мира, а не по Христу» (Кол 2, 8). Как
актуально звучат слова апостола по отношению к различным формам эзотеризма,
которые в наши дни распространяются в некоторых кругах верующих, лишенных
должного критического чувства. Следуяпримеру св. Павла, другие христианские
авторы первых веков, особенно св. Ириней и Тертуллиан, критически высказывались
о культурной тенденции, пытавшейся подчинить истину Откровения толкованию
философов.

38. Итак, сближение христианства и философии не былони мгновенным, ни легким.
Первые христиане считали занятияфилософией и посещение философских школ скорее
помехой,нежели помощью. Свою первую и неотложную задачу они виделив
проповедовании учения о Христе, воскресшем из мертвых, которое следовало
провозгласить каждому, чтобы склонить его к обращению и к просьбе о крещении.
Однако это не значит, что онине заботились о глубоком понимании веры и ее
причин. Совсемнаоборот. Следовательно, обвинения Цельса в адрес христианв
«неграмотности и невежестве» \footnote{Ориген, Contra Celsum, 3, 55: SC 136,
130.}  несправедливы и безосновательны. Причина первоначального пренебрежения
христиан к филосо-

фии кроется в другом. В действительности, чтение Евангелия давало им настолько
исчерпывающий ответ на вопрос о смысле жизни, до этого казавшийся неразрешимым,
что посещение философских школ казалось им чем-то из области прошлого и, в
некотором роде, пережитком.

Это становится еще более очевидным, если мы осознаем, что христианство внесло
вклад в защиту всеобщего права на доступ к истине. Уничтожив расовые,
социальные и связанные с неравноправием полов барьеры, христианство с самого
начала провозгласило равенство всех людей перед Богом. Первым следствием этого
представления было изменение отношения к истине. Так, полностью было отброшено
понятие высшего общества, которому одному лишь, согласно древним, принадлежало
право искать истину. Поскольку доступ к истине является благом, позволяющим
прийти к Богу, все должны иметь возможность пройти этот путь. Существует
множество путей, ведущих к истине; но так как христианская истина обладает
спасительной силой, можно следовать по любому из этих путей, при условии, что
он приведет к конечной цели, т.е. к Откровению Иисуса Христа.

Среди тех, кто приветствовал конструктивный диалог с философской мыслью, хотя и
призывал к осторожному благоразумию, следует упомянуть св. Юстина: хотя он
высоко ценил греческую философию, он твердо и однозначно утверждал, что
христианство является «единственной безошибочной и плодотворной философией»
\footnote{Dialogus cum Tryphone Iudaeo, 8, 1: PG 6, 492.}  .  Климент
Александрийский тоже называл Евангелие «истинной философией»
\footnote{Stromati, I, 18, 90, 1: SC 30, 115.}  и считал философию ближайшим,
после закона Моисея, введением в христианскую веру \footnote{См. там же, I, 16,
80, 5: SC 30, 108.}  и подготовкой к Евангелию \footnote{См. там же, I, 5, 28,
1: SC 30, 65.}  . Поскольку «философия желает той мудрости, которая состоит в
благе души, правоте слова и чистоте жизни, она хорошо подготавливает к мудрости
и делает все, чтобы овладеть ею. Философами же мы называем тех, кто любит
таковую мудрость, которая все творит и всех наставляет, т.е. познание Сына
Божьего» \footnote{Там же, VI, 7, 55, 1-2: PG 9, 277.} .  По мнению Климента,
главной целью греческой философии является не совершенствование или
подтверждение христианской истины, а защита веры: «Учение Спасителя, будучи
Божией силой и Бо-жией премудростью, является всесовершенным и ни в чем не нуж-

дается. Если присоединить к ней греческую философию, истина не станет
действеннее, но так как философия делает тщетными все нападки софистов и
разрывает тайные сети, расставленные истине, мы называем ее оградой и стеной,
окружающей виноградник» \footnote{Там же, I, 20, 100, 1: SC 30, 124.} .

39. По мере того как развивались события, христианские полемисты перенимали
философское познание в строгом смысле. Первым и значительнейшим из примеров,
которые можно найти, является Ориген. Опровергая нападки философа Цельса,
Ориген использует философию Платона, чтобы обосновать свои контраргументы.
Используя многие элементы философии Платона, он начинает формировать
первоосновы христианского богословия. Самопонятие богословия как рационального
рассуждения о Боге еще было связано с греческой культурой, в рамках которой оно
возникло.Например, в философии Аристотеля термином «богословие» обозначалась
наиболее благородная часть и как бы кульминация философских рассуждений. В
свете же христианского Откровения то,что прежде означало учение о божествах,
получило совершенно новый смысл, ибо стало означать рассуждения верующего с
цельюсформулировать истинное учение о Боге. Это постепенно распространяющееся
новое христианское понятие основывалось на философии, но вместе с тем старалось
как-то от нее отмежеваться. История показывает, что учение Платона, воспринятое
христианскимбогословием, претерпело значительные изменения, в особенностив том,
что касалось бессмертия души, обожествления человекаи происхождения зла.

40. В этом процессе христианизации платонизма и неоплатонизма особого внимания
заслуживают Каппадокийские отцы, Дионисий, именуемый Ареопагитом, и, главным
образом, св. Августин. Этот великий западный Учитель Церкви ознакомился с
различными философскими школами, но ни одна не оправдала егонадежд. Когда же
ему была явлена истина христианской веры, оннашел в себе силы совершить
подлинное обращение, к чему его немогли склонить философы, которых он часто
посещал. Вот как онсам описывает мотивы своего шага: «...Я стал предпочитать
католическое учение (манихейскому), поняв, что в его повелении веритьв то, что
нельзя доказать (может быть, доказательство и существует,

но, пожалуй, не для всякого, а может, его и вовсе нет), больше скромности и
правды, чем в опрометчивых насмешках представителей лжеименного знания над
легковерием, после чего они предлагают верить во множество неправдоподобных и
абсурдных вещей, которые не могут доказать» \footnote{Св. Августин,
Confessiones, VI, 5, 7: CCL 27, 77-78.}  . Св. Августин порицал и самих
платоников, о которых чаще всего упоминал в своих произведениях, причем порицал
за то, что, хотя они и знали цель, к которой следует стремиться, но не постигли
путь, ведущий к ней, а именно, Воплощенное Слово \footnote{См. там же, VII, 9,
13-14: CCL 27, 101-102.} . Гиппонский епископ смог создать первый синтез
философских и богословских учений, в котором встретились различные направления
греческой и латинской мысли. В рамках этого синтеза единство знаний, основанное
на библейской мудрости, было подтверждено и усилено глубиной умозрительных
заключений. На протяжении столетий синтез св. Августина оставался самой
совершенной формой философских и богословских рассуждений на Западе. Благодаря
личному опыту и праведному образу жизни, Августин внес в свои труды
многочисленные идеи, которые, будучи связаны с опытом, обеспечили развитие
некоторых философских направлений.

41. Итак, западные и восточные Отцы Церкви различными способами были связаны с
философскими школами. Это не значит, что содержание их учения совпадало с
системами, которые они использовали. Вопрос Тертуллиана: «Что общего между
Афинами и Иерусалимом? А между Академией и Церковью?» \footnote{De
praescriptione hereticorum, VII, 9: SC, 46, 98. «Quid ergo Athenis
etHierosolymis? Quid academiae et ecclesiae?».}  — ярко свидетельствует о
критическом отношении, с каким христианские мыслители с самого начала
рассматривали проблему взаимодействия веры и философии, отмечая как пользу, так
и пределы его. Это не были наивные мыслители. Именно потому, что они глубоко
переживали истины веры, им удалось достичь глубин спекулятивного мышления.
Поэтому неправильно было бы сводить их деятельность лишь к выражению истин веры
в философских категориях. Они сделали намного больше! Они позаботились о том,
чтобы в полном свете явилось то, что было представлено лишь косвенно и в общих
чертах в философии великих мыслителей древности \footnote{См. Священная
Конгрегация по делам христианского воспитания, Инструкция об изучении трудов
Отцов Церкви будущими священниками (10 ноября 1989 г.), 25: AAS 82 (1990),
617-618.}  . Задачей последних, как я уже сказал, было показать, каким образом
разум, освобожденный от внешних оков, может выйти из тупика

мифов, чтобы принять трансцендентное. Итак, освобожденный и правильно
воспитанный разум мог достичь высшего уровня размышлений, создав прочный
фундамент для осознания бытия, трансцендентного сущего и абсолюта.

Именно в этом заключается новизна творений Отцов Церкви. Они полностью приняли
достижения разума, готового принять абсолют, и привили ему богатство,
почерпнутое из Откровения. Эта встреча произошла не только в сфере культур,
одна из которых может восхищаться другой; она произошла в глубине душ и явилась
встречей творения со своим Творцом. Перешагнув предел, к которому он
неосознанно стремился в силу своей природы, человеческий разум сумел достичь
высшего блага и высшей истины в Воплощенном Слове. Что же касается различных
философских систем, то Отцы Церкви смело указывали как на то, что роднило, так
и на то, что отличало их от Откровения. Осознание сходства не мешало им
замечать и различия.

42. В схоластическом богословии роль разума, сформированного в философской
школе, становится более значительной под влиянием интерпретации понятия
«уразумение веры», сформулированного св. Ансельмом. По мнению святого
архиепископа Кентерберийского, примат веры не запрещает разуму вести
самостоятельные поиски, ибо задача разума не заключается в оценке истин веры;
этого он делать не может, так как не способен к этому. Его роль заключается
скорее в поисках смысла, в открытии рациональных доказательств, которые
позволят всем людям каким-то образом осознать истины веры. Св. Ансельм
подчеркивает, что разум должен искать то, что любит: чем сильнее любит, тем
сильнее стремится познать. Кто живет ради истины, тот стремится к определенной
форме познания, которая вызывает в нем все более сильную любовь к тому, что он
постигает, хотя он сам должен признать, что еще не достиг всего желаемого: «Я
сотворен, чтобы лицезреть Тебя, и еще не достиг того, ради чего сотворен»
\footnote{Св. Ансельм, Proslogion, 1: PL 158, 226.}  .  Итак, желание постичь
истину велит разуму стремиться все дальше; разум поражается своим непрестанно
возрастающим возможностям расширять горизонты своего познания. Но в этот момент
разум также может заметить, где заканчивается его путь: «По-моему, должно быть
достаточно для

исследующего прийти к непостижимой вещи, если он путем рассуждения придет к
неколебимому убеждению, что эта вещь действительно существует, хотя и не сможет
проникнуть умом в то, как она существует. (...) Но что так непостижимо, так
неизреченно, как то, что превыше всего? Поэтому если все, что до сих пор
высказывалось о высшей сущности, утверждалось на основании необходимых доводов,
то даже при условии, что интеллект не сможет проникнуть в нее настолько
глубоко, чтобы выразить ее словами, уверенность в ее существовании ничуть не
пошатнется. Ибо, если благодаря приведенному выше рассуждению разум сознательно
постигает непостижимость бытия (rationabiliter comprehendit incomprehensibile
esse), т.е. непостижимость того, каким образом эта высшая мудрость знает
сотворенное ею (...), то кто может объяснить, каким образом она знает или
изрекает себя саму, — о чем человеку дано знать либо лишь немногое, либо вообще
ничего?» \footnote{Св. Ансельм, Monologion, 64: PL 158, 210.} .

Вновь подтверждается фундаментальная гармония между философским познанием и
познанием веры: вера требует, чтобы ее объект постигался разумом; разум,
достигая вершины поисков, признает, насколько необходимо ему то, что являет
вера.

Непреходящая новизна идей св. Фомы Аквинского

43. Особую роль в этом длительном процессе сыграл св. Фома — и не только в
связи с содержанием своей доктрины, но и в связи с диалогом, который он сумел
установить с арабской и иудейской философией того времени. В эпоху, когда
христианские мыслители заново открывали сокровища древней философии и, прежде
всего, труды Аристотеля, заслугой Фомы было то, что он сумел в истинном свете
показать гармонию разума и веры. Поскольку и свет веры, и свет разума
происходят от Бога, аргументировал он, они не могут противоречить друг другу
\footnote{См. св. Фома Аквинский, Summa contra Gentiles, I, VII.} .

Трактуя проблему еще более радикально, св. Фома считает, что природа — истинный
предмет исследований философии — может способствовать пониманию Божественного
Откровения. Следовательно, вера не боится разума, а просит его о помощи и
доверяет ему. Как благодать основывается на природе и совершенствует ее
\footnote{См. Summa Theologiae, I, 1, 8 ad 2: «cum enim gratia non tollat
naturam sed perficiat».}  ,

так вера основывается на разуме и совершенствует его. Разум, просвещенный
верой, освобождается от недостатков и ограничений, источником которых является
грехопадение, и получает необходимую силу, чтобы вознестись к познанию
Триединого Бога. Хотя Ангельский Учитель подчеркивал сверхъестественную природу
веры, он не забывал также о том, какую ценность представляет разумность веры;
наоборот, он умел ее глубоко анализировать и точно определять ее смысл. Ведь
вера некоторым образом является «действием мысли»; человеческий разум не
отказывается от самого себя и не унижает себя, если принимает истины веры;
человек всегда приходит к ним в результате добровольного и осознанного выбора
\footnote{См. Иоанн Павел II, Выступление перед участниками IX
МеждународногоКонгресса, посвященного томизму (29 сентября 1990): Insegnamenti,
XIII, 2(1990), 770-771.} .

Именно по этой причине Церковь всегда считала св. Фому мастером умозаключений и
образцом надлежащего изучения богословия. Я бы хотел привести слова, которые
написал мой предшественник раб Божий Павел VI по случаю семисотлетия со дня
смерти Ангельского Учителя: «Без сомнения, Фома отличался смелостью в поисках
истины, свободой духа в решении новых проблем и интеллектуальной честностью,
присущей тем, которые, не позволяя заражать христианство светской философией,
не отвергают последнюю a priori. Поэтому он вошел в историю христианской мысли
как первопроходец в философии и естествознании. Основой учения, с помощью
которого он, будучи одарен необычайным и почти пророческим гением, разрешил
проблему новых взаимоотношений между разумом и верой, было умение совместить
секулярную природу мира и радикализм евангельских заповедей, благодаря чему он
избежал противоречащего природе отречения от мира и его благ, но и не отошел от
высших непоколебимых принципов сверхъестественного порядка»
\footnote{Апостольское послание Lumen Ecclesiae (20 ноября 1974), 8: AAS 66
(1974), 680.}  .

44. К великим интуитивным идеям св. Фомы относится также та, которая указывает
на роль Святого Духа в процессе роста человеческих познаний до подлинной
мудрости. Уже на первых страницах Суммы теологии \footnote{См. Summa Theologiae
I, 1, 6: «Praeterea, haec doctrina per studium acquiritur.Sapientia autem per
infusionem habetur, unde inter septem dona Spiritus Sancticonnumeratur».}  Фома
Аквинский хотел показать приоритет той мудрости, которая является даром Святого
Духа и вводит в познание Божественной реальности. Его богословие позволяет
понять специфику мудрости в тесной связи с верой и Божественным

познанием. Мудрость постигает благодаря единству природы, основана на вере и
дает правильную оценку, исходя из истины самой веры: «Мудрость как один из
даров Святого Духа отличается от той, которая считается приобретенным свойством
разума. Ибо последняя приобретается благодаря учебе, а та «нисходит свыше»
(Иак. 3, 15). Она также отличается и от веры. Ибо вера принимает Божественную
истину как она есть, а особенностью дара мудрости является то, что она
оценивает в соответствии с Божественной истиной» \footnote{Там же, II—II, 45, 1
ad 2; см. также II—II, 45, 2.}  .

Хотя Ангельский Учитель признает приоритет этой мудрости, он не забывает о
существовании двух других взаимодополняющих форм: философской мудрости, которая
основывается на способности разума изучать действительность в границах,
определенных его природой; а также мудрости богословской, которая основана на
откровении и изучает истины веры, приближаясь к тайне самого Бога.

Будучи твердо убежденным, что «всякая истина, кем бы она ни изрекалась, исходит
от Святого Духа» \footnote{Там же, I—II, 109, 1 ad 1, использовано известное
выражение псевдо-Амвросия, In 1 Cor 12, 3: PL 17, 258.}  , св. Фома возлюбил
истину бескорыстной любовью. Он искал ее везде, где бы она ни появлялась,
стараясь подчеркнуть ее универсальность.  Учительство Церкви отмечает и высоко
ценит его любовь к истине; его мысль — именно потому, что никогда не теряла из
виду объективной, трансцендентной и универсальной истины — достигла «высот,
которых человеческий разум никогда не смог бы себе представить» \footnote{Лев
XIII, энциклика Aeterni Patris (4 августа 1879 г.): AAS 11 (1878-1879), 109.} .
Поэтому его можно по праву назвать «апостолом истины» \footnote{Павел VI,
Апостольское послание Lumen Ecclesiae (20 ноября 1974 г.), 8: AAS66 (1974),
683.}  . Именно потому, что он не колеблясь стремился к истине, ему удалось
реалистически признать ее объективность. Его философия является подлинной
онтологией, а не просто феноменологией.

Трагедия противостояния веры и разума

45. Со времени появления первых университетов богословие вступило в более
тесные контакты с другими видами исследований и научных знаний. Хотя св.
Альберт и св Фома учили о существовании определенной связи между богословием и
философией, они первыми признали, что философия и различные научные дисциплины
должны быть автономны, чтобы проводить исследования

в своих областях. Но начиная с позднего Средневековья это справедливое
разграничение двух областей знаний перешло в пагубное противостояние.
Чрезмерный рационализм некоторых мыслителей привел к радикализации позиций и
появлению философии, автономной по отношению к истинам веры и практически
полностью изолированной от них. В результате этого противостояния появилась
некая неуверенность, распространившаяся и на оценку самого разума. Некоторые
заняли позицию полного скептицизма и агностицизма с целью расширить границы
веры или же, напротив, лишить ее любых рациональных обоснований.

Короче говоря, то, что понималось в патристике и в средневековом богословии как
глубокое единство и являлось источником познания, способного достичь высших
форм теоретического обобщения, было на практике уничтожено системами, которые
защищали рациональное познание, изолированное от веры и являвшееся ее
альтернативой.

46. Экстремальные мнения, которые оказали наиболее сильное влияние, особенно в
западной истории, известны и легко различимы. Нет преувеличения в утверждении,
что развитие философской мысли в период новой истории в основном происходило
при постепенном отдалении от христианского Откровения и даже вступало с ним в
явное противоречие. В прошлом веке этот процесс достиг наивысшей точки.
Некоторые представители «идеализма» различными способами пытались преобразовать
веру и ее истины, даже тайну смерти и воскресения Христа, в диалектические
структуры, доступные рациональному познанию. Этому течению противостояли
различные виды философски проработанного атеистического гуманизма, сторонники
которого считали, что вера вредна и препятствует полноценному развитию разума.
Причем эти течения без колебания принимали статус новых религий, использующих
некоторые привлекательные лозунги, и послужили в социальной и политической
области основой для создания тоталитарных систем, оказавшихся губительными для
человечества.

В сфере естественных наук стало преобладать позитивистское мышление, которое не
только разорвало все связи с христианским видением мира, но, что самое главное,
отказалось от всех метафи-

зических и нравственных понятий. Вследствие этого некоторые ученые, полностью
отказавшись от нравственных ценностей, перестали помещать человеческую личность
и всю ее жизнь в центр своих исследований. Более того, некоторые из них,
осознав, какие возможности открываются благодаря развитию техники, не только
поддались соблазнам рынка, но и поставили своей целью добиться власти демиурга,
чтобы управлять природой и даже самим человеком.

Последствия кризиса рационализма в конце концов привели к возникновению
нигилизма. Для наших современников он в какой-то мере притягателен как
философия небытия. Его сторонники считают, что поиски сами по себе являются
целью, поскольку нет надежды и возможности достичь истины. В интерпретации
нигилистов существование дает лишь совокупность ощущений и опыта, имеющую
преходящий характер. Нигилизм лежит в основе распространенного в наши дни
мнения, что не стоит брать на себя постоянных обязательств, поскольку все
изменчиво и временно.

47. С другой стороны, не следует забывать, что роль философии в современной
культуре изменилась. Она потеряла статус мудрости и универсального знания и
постепенно превратилась в одну из многочисленных форм интеллектуальной
деятельности; более того, некоторые считают ее вовсе излишней. Другие формы
рационального познания при этом приобретают все большее значение, подчеркивая
второстепенный характер философского знания. Вместо того чтобы служить
созерцанию истины, а также поискам конечной цели и смысла жизни, эти формы
рационального познания используются — или, во всяком случае, могут
использоваться — как «инструментальный разум» для достижения утилитарных целей,
личной выгоды и власти.

Уже в первой моей энциклике я указал, какая серьезная опасность возникает,
когда человек придает этому пути абсолютный характер: «Создается впечатление,
что современному человеку постоянно угрожает то, что он создает, то есть плоды
его трудов и, в еще большей степени, плоды его разума и устремлений его воли.
Плоды многообразной деятельности человека часто очень быстро и непредвиденным
образом подвергаются «отчуждению», то есть ока-

зываются отобранными у производителя, при этом они зачастую прямо или косвенно
обращаются против самого же человека. Кажется, что в этом и состоит взятая в ее
самом обширном и универсальном измерении основная суть трагедии современного
человеческого существования. В результате человека все больше охватывает страх.
Он опасается, что плоды его труда могут обернуться против него — конечно, не
все они, и даже не большая их часть, — а некоторые, но именно те, в которых
отразились лучшие стороны его гения и его способности к творчеству»
\footnote{Энциклика Redemptor hominis (4 марта 1979 г.), 15: AAS 71 (1979),
286.}  .

После этих преобразований в культуре некоторые философы, отказавшись искать
истину ради нее самой, выбрали в качестве единственной цели своих поисков
субъективную уверенность или практическую пользу. В результате они упустили из
виду истинное достоинство разума, который в результате лишился способности
познавать истину и искать абсолют.

48. В последнем периоде истории философии наблюдается все возрастающее
размежевание веры и философствующего разума. Разумеется, что при внимательном
изучении философских трудов даже тех авторов, которые способствовали разделению
веры и разума, в них можно обнаружить ценные зерна мысли, которые — при
условии, что они будут приняты и развиты праведным сердцем и разумом — могут
открыть путь к истине. Такие зерна мысли содержатся, к примеру, в глубоком
анализе восприятия и опыта, воображения и бессознательного, личности и
интерсубъективности, свободы и ценности, времени и истории. Мысль о смерти тоже
может стать для каждого философа суровым призывом обрести истинный смысл своей
жизни. Но, несмотря на это, в настоящее время соотношение веры и разума требует
внимательного изучения, поскольку взаимосвязь разума и веры нарушена, что
приводит к обеднению обеих сторон. Разум, лишенный помощи Откровения, обречен
на блуждание окольными путями, в результате чего он подвергается опасности
потерять из виду конечную цель. Вера, лишенная поддержки разума,
сосредоточилась на чувствах и переживаниях, поэтому возникает опасность, что
она утратит универсальный характер. Ошибочно мнение, что вера сильнее
воздействует на слабый разум; наоборот, в этом случае она подвергается
серьезной

опасности, поскольку может быть сведена к уровню мифов или предрассудков.
Аналогично, если разум сталкивается с незрелой верой, ему не хватает стимулов,
чтобы сосредоточить внимание на глубине и новизне бытия. Поэтому пусть не
покажется неуместным мой серьезный и решительный призыв восстановить единство
веры и разума, которое позволит им действовать в соответствии с их природой,
сохраняя взаимную автономию. Глубине веры должно соответствовать дерзновение
разума.

\section{Суждения Учительства Церкви в области философии}

Рассудительность Учительства Церкви как служение истине 

49. Церковь не провозглашает собственную философию и официально не поддерживает
ни одно из философских направлений в противовес всем остальным \footnote{См.
Пий XII, Энциклика Humani generis (12 августа 1950): AAS 42 (1950), 566.}  .
Такая сдержанная позиция вызвана тем, что философия, даже если она соотносится
с богословием, должна использовать собственные методы и подчиняться собственным
правилам; в противном случае нельзя гарантировать, что она будет направлена к
истине и стремиться к ней таким образом, чтобы это мог контролировать разум.
Если бы философия не действовала в свете разума и в соответствии с собственными
принципами и специфическим методом, в ней было бы мало пользы. Глубинное
основание автономии, которой пользуется философия, заключается в том, что разум
в силу своей природы стремится к истине и, кроме того, наделен необходимыми
средствами для ее достижения. Разумеется, что философия, осознавая свое
«первоначальное строение», будет также соблюдать требования и утверждения,
характерные для богооткровенной истины.

Однако история показывает, что философская мысль, особенно в новой истории,
часто сворачивает с пути истинного и заблужда-

ется. Задача Учительства Церкви не заключается в том, чтобы восполнять пробелы
в несовершенных философских умозаключениях. Но оно, безусловно, должно
решительно вмешиваться, если спорные философские тезисы угрожают правильному
пониманию истин Откровения, а также если становятся популярными ложные и
односторонние мнения, которые распространяют серьезные заблуждения, нарушающие
простоту и искренность веры народа Божьего.

50. В силу своего авторитета и в свете веры Учительство Церквиможет и должно
критически оценивать философию и взгляды, которые противоречат христианскому
учению \footnote{См. I Ватиканский Собор, Первая догматическая конституция о
ЦерквиХристовой Pastor Aeternus: DS 3070; II Ватиканский Собор,
Догматическаяконституция о Церкви Lumen gentium, 25 и далее.}  . Задачей
Учительства, прежде всего, является указывать, какие философские предпосылки и
выводы противоречат истине Откровения, а тем самымформулировать требования,
которые следует предъявлять философии с точки зрения веры. Кроме того,
поскольку в процессе развития философии сформировались различные школы
мышления,многообразие этих направлений обязывает Учительство Церкви выражать
свое мнение о соответствии или несоответствии основныхконцепций, которыми
руководствуются эти школы, требованиямслова Божьего и принципам богословских
рассуждений.

Церковь обязана указывать, что именно в данной философской концепции может
противоречить вере. Действительно, многие философские размышления о Боге, о
человеке и его свободе, о нравственных нормах поведения требуют оценки со
стороны Церкви, поскольку касаются истин Откровения, вверенных ее защите. Когда
мы, епископы, даем такую оценку, нам надлежит быть «свидетелями истины»,
смиренно, но стойко исполняющими служение, важность которого обязан оценить
каждый философ, ибо оно является основой для «здравого разума», т.е. разума,
правильно размышляющего об истине.

51. Однако не следует понимать такой подход Церкви как лишьнегативную позицию,
как то, что Учительство стремится исключить или ограничить какие-либо
инициативы. Наоборот, Учительство Церкви должно, прежде всего, порождать,
поддерживать и стимулировать философские исследования. Впрочем, сами
философылучше всего сознают необходимость самокритики, исправления

возможных ошибок и выхода за слишком узкие рамки, которыми ограничены
философские размышления. Следует помнить, что существует только одна истина,
хотя она и выражается в формах, которые несут на себе печать истории, и, кроме
того, созданы человеческим разумом, затемненным и ослабленным грехом. Из этого
следует, что ни одна философия, созданная в ходе истории, не может обоснованно
утверждать, что она выражает всю истину или же полностью объясняет реальность
человека, мира и отношений человека с Богом.

Кроме того, в наши дни, когда появляются все новые философские системы, методы,
понятия и аргументы, иногда очень подробно разработанные, возникает
необходимость подвергнуть их критической оценке в свете веры. Это нелегкая
задача, ибо бывает трудно даже определить естественные и неотъемлемые
способности разума, учитывая его внутреннюю, а также исторически обусловленную
ограниченность, и тем более трудно иногда бывает отличить в философских
концепциях ценные и плодотворные с точки зрения веры мысли от ошибочных и
опасных идей. Церковь, разумеется, знает, что сокровища премудрости и ведения
сокрыты во Христе (Кол 2, 3); поэтому в своих высказываниях она старается,
поощряя философские исследования, не закрывать путь, который ведет к познанию
тайны.

52. Не только в последнее время Учительство Церкви высказывает свое мнение о
некоторых философских мнениях. Достаточно вспомнить, например,
провозглашавшиеся на протяжении веков определения по поводу учения о
предсуществовании душ \footnote{См. Константинопольский Синод: DS 403.}  , а
также различных проявлений идолопоклонства и суеверного эзотеризма в
астрологических предсказаниях \footnote{См. I Толедский Собор: DS 205; I
    Брагский Собор: DS 459-460; Сикст V, булла Coeli et terrae Creator (5
    января 1586): Bullarium Romanum 4/4, Romae 1747,176-179; Урбан VIII,
Inscrutabilis iudiciorum (1 апреля 1631): Bullarium Romanum6/1, Romae 1758,
268-270.}  .  Достойны упоминания и более систематические сочинения,
направленные против некоторых тезисов латинского аверроизма, противоречащих
христианской вере \footnote{См. Вьеннский Собор, Декрет Fidei catholicae: DS
902; V ЛатеранскийСобор, булла Apostolici regiminis: DS 1440.}  .

Начиная со второй половины прошлого века участились высказывания Учительства
Церкви, поскольку в тот период многие католики считали своим долгом
противопоставить различным течениям новой мысли свою собственную философию. В
этой ситуации Учительство Церкви обязано было следить, чтобы эта философия не

принимала ложные и негативные формы. Его критика, симметрично направленная в
двух противоположных направлениях, с одной стороны, касалась фидеизма
\footnote{См. Theses a Ludovico Eugenio Bautain iussu sui Episcopi subscriptae
    (8 сентября 1840 г.): DS 2751-2756; Theses a Ludovico Eugenio Bautain ex
    mandato S.Cong.Episcoporum et Religiosorum subscriptae (26 апреля 1844 г.):
    DS 2765-2769.}  и радикального традиционализма \footnote{См. Св.
Конгрегация Индекса, декрет Theses contra traditionalismum Augustini Bonnetty
(11 июля 1855 г.): DS 2811-2814.}  в связи с их неверием в природные
возможности разума, а с другой — рационализма61 и онтологизма62 , которые
приписывали естественному разуму знания, которые можно постичь лишь в свете
веры.  Конструктивное содержание этой полемики формально выражено в
догматической конституции Dei Filius, в которой впервые Вселенский Собор, а
именно, I Ватиканский Собор, официально высказался о соотношении веры и разума.
Это учение Собора оказало сильное и положительное влияние на философские
искания многих верующих и до сих пор является надежным образцом правильных
христианских рассуждений в этой области.

53. В своих высказываниях Учительство Церкви уделяло внимание не столько
отдельным философским тезисам, сколько необходимости естественного познания —
то есть, в сущности, философского познания — в размышлениях о вере. I
Ватиканский Собор, выразив в виде синтеза и торжественно подтвердив учение,
которое ранее являлось обычным систематическим папским учительством,
адресованным верующим, указал, насколько неразрывны и в то же время четко
разграничены естественное познание Бога и Откровение, т.е. разум и вера. Собор
начал с фундаментальной предпосылки, лежащей в основе самого Откровения, а
именно, что возможно естественное познание существования Бога, источника и цели
всего сущего63 ; а закончил торжественным заявлением, которое я уже цитировал:
«Существуют два вида познания, которые имеют не только разные источники, но и
объекты»64 . Следовательно, необходимо было подтвердить, вопреки всем видам
рационализма, отличие тайн веры от философских построений, а также то, что
первые предшествуют последним и превосходят их; с другой стороны, следовало
противостоять ложным идеям фидеизма и напомнить о единстве истины, то есть о
том положительном вкладе, которое рациональное познание может и должно вносить
в постижение веры: «Хотя вера превосходит разум, нет и не может быть
разногласий между верой и разумом, ибо тот же самый Бог, который открывает
тайны и сообщает веру, даровал человеческой душе свет разума;

но Бог не может отречься от Самого Себя и истина не может противоречить
истине»65 .

54. В нашем веке Учительство Церкви тоже неоднократно высказывалось на эту
тему, предостерегая от искушений рационализма. В связи с этим следует упомянуть
труды Папы Пия X, который указывал, что модернизм основывается на философских
тезисах, являющихся проявлением феноменализма, агностицизма и имманентизма66 .
Не следует также забывать о другом важном факте, а именно о том, что
католичество отвергает марксистскую философию и атеистический коммунизм67 .

В свою очередь Папа Пий XII предостерегал в энциклике Humani generis от
ошибочных идей, основанных на тезисах эволюционизма, экзистенциализма и
историзма. Понтифик ясно указал, что эти учения не были созданы и провозглашены
богословами, но родились «вне овчарни Христовой»68 ; он же добавил, что эти
заблуждения не следует попросту отвергать, но необходимо их критически
оценивать: «Католические философы и богословы, важной задачей которых является
защита Божественной и человеческой истины, а также ее укоренение в сердцах
людей, не могут игнорировать или пренебрегать мнениями, которые в большей или
меньшей степени уклоняются от истинного пути. Наоборот, они должны их тщательно
изучить, поскольку невозможно излечить от болезней, если их прежде не
исследовать, а также потому, что даже в ложных утверждениях иногда кроется
крупица истины, и, наконец, эти заблуждения побуждают наш разум более
внимательно изучать и размышлять над некоторыми философскими и богословскими
истинами»69 .

В недалеком прошлом и Конгрегация вероучения, выступая от имени Римского
епископа70 , была вынуждена вмешаться, чтобы вновь обратить внимание на
опасность, связанную с бесконтрольным принятием некоторыми богословами,
придерживающимися т.н. «теологии освобождения», тезисов и методологии
марксистского толка71 .

Итак, в прошлом Учительство Церкви при различных обстоятельствах многократно
выносило суждения по философским вопросам. Вклад моих уважаемых
предшественников является ценным достоянием, о которым нельзя забывать.

55. Изучая современную ситуацию, мы замечаем, что вновь появляются проблемы
минувших дней, но уже в иной форме. Мы уже имеем дело не только с вопросами,
которые интересуют отдельных людей или определенные круги, а со взглядами,
которые распространены столь широко, что в определенной степени формируют
мышление всех людей. Например, к ним относится недоверчивое отношение к разуму,
заметное в новейших течениях философской мысли. В связи с этим во многих кругах
поговаривают о «гибели метафизики», требуют, чтобы философия ограничилась более
скромными задачами, такими как интерпретация фактов или рассуждения об
определенных областях человеческих знаний или их структурах.

В самом богословии снова возникают прежние соблазны. Например, в некоторых
современных богословских школах вновь проявляется некий рационализм, например,
когда мнения, признанные философски обоснованными, считаются обязательными в
богословских исследованиях. Это чаще всего происходит в тех случаях, когда
богослов, не обладающий достаточными философскими познаниями, бесконтрольно
поддается влиянию утверждений, которые уже вошли в обиход и в массовую
культуру, но не имеют под собой необходимых рациональных обоснований72 .

Небезопасно также возвращение к идеям фидеизма, который не признает, что
рациональные знания и философские размышления обусловливают понимание веры, и
даже саму возможность веры в Бога. Распространенным проявлением фидеистических
тенденций в наше время является «библеизм», который хотел бы сделать чтение
Священного Писания и его экзегезу единственной достоверной отправной точкой. Он
доходит до того, что отождествляет слово Божие только со Священным Писанием и
таким образом отвергает учение Церкви, которое было подтверждено II Ватиканским
Собором. Конституция Dei Verbum напоминает, что Слово Божие присутствует как в
священных текстах, так и в Предании73 , и торжественно провозглашает:
«Священное Предание и Священное Писание составляют единый священный залог слова
Божия, вверенный Церкви; придерживаясь его, весь святой народ, со своими
пастыря-

ми, пребывает постоянно в учении апостолов»74 . Итак, Священное Писание не
является единственным достоверным источником для Церкви, ибо источником
«наивысшего правила ее веры»75 является единство, которое Святой Дух установил
между Преданием, Священным Писанием и Учительством Церкви, так связав и
соединив их между собой, что они не могут существовать отдельно76 .

Не следует также пренебрегать опасностью, которой подвергается тот, кто
пытается постигать истины Священного Писания с помощью только одной
методологии, забывая о необходимости более полной экзегезы, которая позволяет
ему вместе со всей Церковью целиком раскрыть смысл текстов. Тот, кто занимается
изучением Священного Писания, должен всегда помнить, что различные
герменевтические методологии основаны на определенных философских концепциях,
поэтому их следует внимательно изучить, прежде чем применять для анализа
священных текстов.

Другие проявления скрытого фидеизма — это недоверие к теоретическому богословию
и пренебрежительное отношение к классической философии, понятия которой помогли
рационально истолковать истины веры и даже были использованы в догматических
формулах. Блаженной памяти Папа Пий XII предостерегал от последствий такого
разрыва с философской традицией и отказа от традиционной терминологии77 .

56. Вообще в настоящее время многие с недоверием относятся к утверждениям
общего и абсолютного характера, особенно те, которые являются приверженцами
теории, гласящей, что истина является результатом договоренности, а не
признания разумом объективной реальности. Разумеется, в мире, разделенном на
множество специфических областей знаний, становится труднее найти полный и
окончательный смысл жизни, который ищет традиционная философия. Тем не менее, в
свете веры, которая видит этот конечный смысл в Иисусе Христе, я не могу не
призвать философов, как христианских, так и нехристианских, поверить в
способности человеческого разума и не ставить слишком скромных целей перед
философскими размышлениями. История тысячелетия, которое близится к концу, учит
нас, что следует идти именно этим путем: стремиться к окончательной истине,
ревностно ее искать и смело от-

крывать новые пути. Сама вера торопит разум, чтобы он вырвался из плена
самоизоляции и не колеблясь шел на риск в поисках красоты, добра и истины.
Итак, вера становится решительной и убедительной защитницей философии.

Отношение Церкви к изучению философии

57. Однако Учительство Церкви не ограничивалось указаниемошибок и искажений в
философских доктринах. Оно столь же последовательно старалось напоминать
основные принципы, на которых должно быть основано подлинное обновление
философскоймысли, и одновременно указывало конкретные пути поисков.В этих
обстоятельствах опубликование энциклики Aeterni PatrisПапы Льва XIII стало
поистине переломным событием в жизниЦеркви. До настоящего времени это был
единственный папский документ такого ранга, полностью посвященный философии.
Этотвыдающийся Папа развил в нем учение Отцов I Ватиканского Собора о
соотношении веры и разума, показав, что философскаямысль имеет фундаментальное
значение для веры и богословскихзнаний78 . Хотя прошло более ста лет, многие
рекомендации из этойэнциклики не потеряли актуальности, как с практической,
таки с педагогической точки зрения; это, прежде всего, касается тезиса о
несравненной ценности философии св. Фомы. Папа Лев XIIIсчитал, что повторное
обращение к философии Ангельского Учителя — это наилучший путь к тому, чтобы
вновь привести философские размышления в соответствие с требованиями веры. «Св.
Фома, — писал Папа, — как это и подобает, четко разграничивает веру и разум,
одновременно связывает их узами взаимной дружбы,обеспечивает соблюдение их прав
и защищает их достоинство»79 .

58. Известно, к каким положительным результатам привел призыв Папы. Изучение
идей св. Фомы и других схоластических авторов было начато с новым рвением.
Значительно активизировалисьисторические исследования, что привело к открытию
сокровищсредневековой мысли, до сих пор, в основном, неизвестных, и
квозникновению новых школ последователей томизма. Благодаряиспользованию
исторической методологии были существенно до-

полнены знания о трудах св. Фомы, а многие ученые смогли смело использовать
традиции томизма в дискуссии о философских и богословских проблемах своего
времени. Наиболее влиятельные богословы нашего века, размышления и искания
которых вдохновили Отцов II Ватиканского Собора, были воспитаны на этом
движении возрождения философии томизма. Благодаря этому в XX веке Церковь
располагала многочисленной группой творческих мыслителей, воспитанных в духе
идей Ангельского Учителя.

59. Томистское и неотомистское обновление не было единственным проявлением
возрождения философской мысли в культурес христианскими традициями. Уже до
призыва Папы Льва XIIIи независимо от него появилось много католических
богословов,которые обратились к современным течениям мысли и,
используясобственную методологию, создали философские труды непреходящей
ценности, оказавшие огромное влияние на умы людей. Одниразработали необычайно
смелый синтез, который ничем не уступал великим идеалистическим системам;
другие заложили основыэпистемологии для нового описания веры в свете
обновленной концепции совести; иные, в свою очередь, создали философию,
которая, исходя из анализа имманентной действительности, открывалапуть к
трансцендентному; некоторые попытались совместить требования веры с принципами
феноменологической методологии.Итак, исходя из разных предпосылок, постоянно
создавались новыевиды философских отвлеченных умозаключений, которые продолжили
славную традицию христианской мысли, основаннойна единстве веры и разума.

60. Отцы II Ватиканского Собора также разработали богатоеи плодотворное учение,
касающееся философии. Особенно в связис настоящей энцикликой я должен
напомнить, что целая глава Конституции Gaudium et spes является своеобразным
компендиумомбиблейской антропологии, которая также вдохновляет философов.В этом
документе говорится о ценности человека, созданного по образу и подобию
Божиему, обосновывается его достоинство и превосходство над остальными
творениями, показаны трансцендентные способности его разума80 . В Gaudium et
spes также рассматривается проблема атеизма и убедительно указаны причины этого

философского заблуждения, особенно в контексте неотъемлемого достоинства
человека и его свободы81 . Несомненно, глубоким философским смыслом наполнены
слова, выражающие главную мысль этого документа, на которые я сослался в моей
первой энциклике Redemptor hominis и которые являются одной из неизменных
отправных точек моего учения: «Действительно, тайна человека истинно освещается
только в тайне воплотившегося Слова, ибо Адам, первый человек, был образом
грядущего (Рим 5, 14), то есть Христа. Христос, новый и последний Адам, в самом
явлении тайны Отца и Его любви, в полноте открывает человеку его самого и ясно
показывает ему его наивысшее призвание»82 .

Отцы Собора также обсудили проблему изучения философии будущими священниками;
их рекомендации можно использовать более широко для всего христианского
образования: «Философские дисциплины должны преподаваться так, чтобы
воспитанники направлялись к приобретению прочного и стойкого знания о человеке,
мире и Боге, опираясь на вечное ценное философское наследие, при этом учитывая
результаты философских исследований новейшего времени»83 .

В других документах Учительства Церкви напоминаются и развиваются эти указания,
вызванные заботой о всесторонней философской подготовке, прежде всего, тех, кто
собирается изучать богословие. Со своей стороны, я неоднократно подчеркивал
значение философской подготовки для тех, кто в ходе пастырской деятельности
должен будет решать проблемы современного мира и понимать причины определенного
поведения, чтобы правильно на него реагировать84 .

61. Если снова и снова я считал необходимым вновь возвращаться к этому вопросу,
а именно, к подтверждению ценности идей Ангельского Учителя, настоятельно
рекомендуя изучение его наследия, то это было вызвано тем, что соответствующие
рекомендации Учительства Церкви не всегда выполнялись с должным усердием. После
II Ватиканского Собора во многих католических учебных заведениях наблюдался
регресс в этой области, что было вызвано ослаблением внимания не только к
схоластической философии, но и к философии вообще. С удивлением и сожалением я
замечаю,

что многим богословам свойственно это пренебрежение, оказываемое философии.

Такое негативное отношение вызвано различными причинами. Прежде всего, следует
упомянуть неверие в способности разума, характерное для большинства концепций
современной философии, которая отказывается от метафизических размышлений над
извечными вопросами человека и сосредоточивается на частных и узких проблемах,
иногда чисто формальных. Следует также обратить внимание на неправильное
отношение к так называемым «гуманитарным наукам». На II Ватиканском Соборе
неоднократно подтверждалась ценность научных исследований, имеющих целью
глубокое познание тайны человека85 . Призыв к богословам изучать эти науки и по
мере возможности использовать их надлежащим образом в своих поисках не должен
восприниматься как косвенное разрешение отодвинуть философию на второй план или
совершенно отказаться от нее в процессе пастырского воспитания и в «подготовке
к вере» (praeparatio fidei). Наконец, не стоит забывать о вновь возникшем
интересе к проблеме инкультурации веры. Опыт, прежде всего, молодых Церквей
позволил обнаружить, наряду с замечательными формами познания, наличие
различных проявлений народной мудрости, которая является подлинным культурным и
традиционным наследием. Однако исследованию традиций должно сопутствовать
изучение философии. Такие исследования позволят обнаружить ценные элементы
народной мудрости и создать необходимые узы между ними и проповедью Евангелия86
.

62. Я хочу решительно подтвердить, что философия является основным и
необходимым элементом в программе изучения богословия и в подготовке будущих
священников. Не случайно в программе семинарий богословским курсам предшествует
начальный этап, который посвящен, прежде всего, изучению философии. Такой
порядок, подтвержденный на V Латеранском Соборе87 , восходит своими корнями к
опыту, накопленному в Средние века, когда была обнаружена необходимость
соединить в гармоническое целое философские и богословские знания. Такой подход
к изучению богословия в значительной степени формировал, облегчал и
поддерживал, по крайней мере, косвенно, развитие философии нового времени.

Характерным примером является влияние труда «Метафизические рассуждения»
Франсиско Суареса, изучавшегося даже в университетах лютеранской Германии.
Отказ же от этой методологии привел к серьезному ущербу как в деле воспитания
священников, так и для самих богословских исследований. В качестве примера
достаточно упомянуть об отсутствии интереса к современной мысли и культуре, что
приводит к прекращению каких бы то ни было форм диалога или к нерассудительному
принятию любых философских воззрений.

Я глубоко убежден, что эти трудности будут преодолены благодаря разумно
поставленному философскому и богословскому образованию, которое никогда не
должно исчезнуть в Церкви.

63. В силу вышеизложенных причин я посчитал необходимым подтвердить в настоящей
энциклике, что Церковь проявляет живой интерес к изучению философии, так как
существуют прочные узы, связывающие труд богослова с философскими поисками
истины. Поэтому Учительство Церкви обязано стимулировать развитие философии,
что ни в коей мере не противоречит вере. Моя задача — показать некоторые
принципы и отправные точки, которые я считаю непременным условием установления
гармоничной и плодотворной связи между богословием и философией. В их свете
можно будет более ясно увидеть, должно ли богословие поддерживать связь с
различными современными философскими системами и взглядами, и если да, то каким
образом.

\section{Взаимодействие богословия и философии}

Вероучение Церкви и требования философского разума

64. Слово Божие обращено к каждому человеку, во все времена и во всех уголках
земли, человек же по своей природе является философом. В свою очередь,
богословие научными методами осознан-

но развивает понимание слова Божия в свете веры, поэтому не может обойтись без
взаимодействия с философскими системами, создававшимися на протяжении столетий,
в связи с определенными методами, которые она использует, а также стоящими
перед ней задачами. Я не намерен указывать богословам конкретные методологии,
ибо это не относится к компетенции Учительства Церкви; скорее, мне хочется
напомнить о некоторых особых задачах богословия, для осуществления которых
необходимо обратиться к философской мысли, учитывая природу явленного в
Откровении слова.

65. Структура богословия как вероучения сформирована двумяметодологическими
принципами: слышание веры (auditus fidei) ипонимание веры (intellectus fidei).
В свете первого из них богословие принимает истины Откровения, по мере того как
СвященноеПредание, Священное Писание и Учительство Церкви их истолковывают88 .
Применяя второй принцип, богословие старается выполнить специфические
требования разума с помощью теоретическихрассуждений.

Философия вносит свойственный ей вклад в сферу приготовления к истинному
слышанию веры, рассматривая структуру познания и средства общения людей, в
особенности различные формы и функции языка. Важен также вклад философии в
правильное понимание церковного Предания, решений Учительства Церкви и
высказываний великих богословов, поскольку в них часто встречаются понятия и
структуры мышления, заимствованные из определенной философской традиции. В этом
случае богослов должен не только объяснить понятия и термины, используемые
Церковью в своих рассуждениях и учении, но и глубоко познать философские
системы, которые могли повлиять на понятия и терминологию, чтобы благодаря
этому создать правильные и связные интерпретации.

66. Что же касается понимания веры, прежде всего следует отметить, что
Божественная истина, «явленная нам в Священном Писании и верно объясненная в
учении Церкви»89 , сама по себе понятна, ибо обладает такой логической связью,
что является подлинной областью знаний. Понимание веры объясняет эту истинуне
только благодаря тому, что демонстрирует логические и поня-

тийные структуры, из которых складывается учение Церкви, но, прежде всего,
благодаря тому, что обращает внимание на содержащиеся в них спасительные
истины, предназначенные для отдельного человека и всего человечества. Все эти
утверждения позволяют верующему познать историю спасения, венцом которой
является Иисус Христос и Его Пасхальная тайна. Христианин приобщается к этой
тайне, выражая послушание веры.

В свою очередь, догматическое богословие должно уметь выразить универсальный
смысл тайны Триединого Бога и домостроительства спасения как в виде
повествования, так и, прежде всего, в виде аргументов. Оно должна это сделать с
помощью критически сформированных понятий, которые ясны всем, ибо без участия
философии невозможно объяснить такие богословские вопросы, как, например, язык,
описывающий Бога, личностные отношения в лоне Троицы, созидательное действие
Бога в мире, связь Бога и человека, сущность Христа, который является истинным
Богом и истинным человеком. То же самое можно сказать о различных вопросах из
области нравственного богословия, где используются следующие понятия:
нравственный закон, совесть, свобода, личная ответственность, вина и т.п.,
определения которым даются в сфере философской этики.

Поэтому необходимо, чтобы разум верующего обрел естественное, истинное и
адекватное познание сотворенной действительности, мира и человека, которое
также относится к Божественному Откровению; тем более разум должен уметь
выразить эти знания с помощью понятий и аргументов. Следовательно,
теоретическое догматическое богословие предполагает и дополняет определенную
философию человека, мира или, глубже, самого «бытия», основанную на объективной
истине.

67. Как дисциплина, задачей которой является обоснование веры (см. 1 Петр 3,
15), фундаментальное богословие должно пытаться обосновать и объяснить связь
веры с философскими рассуждениями. Уже Отцы I Ватиканского Собора, ссылаясь на
учение св. апостола Павла (см. Рим 1, 19-20), обратили внимание на факт, что
существуют истины, доступные естественному, а значит, и философскому познанию.
Постижение их является обязательным условием принятия Божественного Откровения.
Изучая Откровение

и его достоверность, а одновременно соответствующий акт веры, фундаментальное
богословие должно показать, как в свете познания, просвещенного верой, можно
увидеть определенные истины, которые разум уже начинает постигать в результате
самостоятельных поисков. Откровение придает этим истинам законченный смысл,
направляя их к богатствам богооткровенных тайн, где они обретают свою конечную
цель. В качестве примера можно привести естественное познание Бога, возможность
отличить Божественное Откровение от других явлений или признать его
достоверность, способность правдиво и понятно описать человеческим языком также
то, что находится за пределами человеческого опыта. Все эти истины убеждают
разум, что действительно существует путь, подготавливающий его к вере, которым
он может прийти к принятию Откровения, не нарушая свои принципы или собственную
автономию90 .

Подобным образом фундаментальное богословие также должно показать внутреннее
соответствие между верой и необходимостью ее постижения разумом, который
способен дать согласие на это с величайшей свободой. Благодаря этому вера
сможет «во всей полноте указать путь разуму, искренне ищущему истину. Итак,
вера, дар Божий, хотя и не основана на разуме, не может без него обойтись;
одновременно разум понимает, что должен опираться на веру, чтобы увидеть
горизонты, которых ему не достичь своими силами»91 .

68. Нравственное богословие, быть может, еще больше нуждается в помощи
философии, ведь в Новом Завете человеческая жизнь в меньшей степени
регулируется правилами, чем в Ветхом Завете. Жизнь в Святом Духе ведет верующих
к свободе и ответственности, которые выходят за рамки самого Завета. Тем не
менее, в Евангелии и апостольских посланиях содержатся как общие принципы
христианской жизни, так и подробные предписания. Чтобы руководствоваться ими в
конкретных случаях в личной и общественной жизни, христианин должен уметь в
полном объеме использовать свою совесть и способность к познанию. Другими
словами, это означает, что нравственное богословие должно обращаться к
адекватному философскому видению, как в том, что касается природы человека

и общества, так и применительно к общим принципам, которые управляют
нравственным выбором.

69. Можно возразить, что в нынешнюю эпоху богослов долженобращаться не столько
к философии, сколько к другим областям человеческих знаний, например, к
истории, а особенно, к естественным наукам, необычайный прогресс которых в
последнее времявызывает всеобщее удивление. В свою очередь, в связи с
повышенным вниманием к инкультурации веры, некоторые считают, чтобогословие
должно чаще обращаться к традиционным формам мудрости, а не к философии
греческого происхождения, именуемой«европоцентрической». Другие же, исходя из
ошибочной концепции плюрализма культур, попросту оспаривают ценность
философского наследия, принятого Церковью.

В этих взглядах, которые, впрочем, были приняты во внимание II Ватиканским
Собором92 , заключена определенная доля истины. Во многих случаях обращение к
естественным наукам может оказаться полезным, ибо позволяет получить более
полную информацию о предмете изучения, однако не следует забывать, что в данном
случае необходимо посредничество чисто философского познания, которое имело бы
критический и одновременно универсальный характер: такое познание требуется,
помимо прочего, для плодотворного взаимодействия культур. Хочу прежде всего
обратить внимание на то, что нельзя останавливаться на единичных, частных
случаях, пренебрегая основной задачей, которой является раскрытие
универсального характера истин веры. Не следует также забывать, что особый
вклад философской мысли заключается в том, что она позволяет понять — как в
различных концепциях жизни, так и в культурах — «не то, что думают люди, а
какова объективная истина»93 . Для богословия полезно не знание различных
мнений, но лишь сама истина.

70. Тема связи с культурами заслуживает отдельного обсуждения, которое, однако,
не будет исчерпывающим по причине трудностей как философского, так и
богословского плана. С самого начала проповеди Евангелия Церковь сталкивалась с
различнымикультурами, что иногда приводило к конфронтации. Христос
велелученикам провозглашать явленную Им истину везде, «даже до края

земли» (Деян 1, 8). В результате члены христианской общины очень быстро
убедились во вселенском характере проповедуемого учения и столкнулись с
препятствиями, вызванными культурными различиями. Отрывок из послания св.
апостола Павла к Ефесянам позволяет понять, каким образом первая община
пыталась решить эту проблему. Апостол пишет: «А теперь во Христе Иисусе вы,
бывшие некогда далеко, стали близки Кровью Христовой. Ибо Он есть мир наш,
соделавший из обоих одно и разрушивший стоявшую посреди преграду» (Еф 2,
13-14).

В свете этого отрывка наши размышления охватывают более широкую область и
рассматривают также перемены, которые происходили в язычниках после принятия
веры. Перед лицом богатства спасения, совершенного Христом, рушатся барьеры,
разделяющие разные культуры. Обещание, данное Богом в Иисусе Христе, обретает
универсальное измерение: оно уже не заключено в рамки одного народа, языка и
традиции, но дается всем как наследие, которым может пользоваться каждый. Все
люди из разных стран, имеющих свои традиции, призваны объединиться во Христе в
единую семью детей Божиих. Сам Христос обеспечивает возможность того, чтобы два
народа стали «одно». Те, которые были «далеко» друг от друга, стали «близки»
благодаря новой реальности, установленной Пасхальной тайной. Иисус разрушает
барьеры разделения и создает неизвестное доселе и совершенное единство через
приобщение к Его тайне. Это единство настолько глубоко, что Церковь может
сказать вслед за св. апостолом Павлом: «Итак, вы уже не чужие и не пришельцы,
но сограждане святым и домашние Богу» (Еф 2, 19).

В этом столь простом утверждении заключена великая истина: в результате встречи
веры с различными культурами на практике родилась новая реальность. Если
культуры глубоко укоренились в человеческой природе, для них характерна
типичная для человека готовность принять универсальное измерение и
трансцендентное. Следовательно, они представляют собой разные пути, ведущие к
истине, и, несомненно, полезны для людей, ибо открывают ценности, благодаря
которым наша жизнь становится более человечной94 . Кроме того, если культуры
обращаются к ценностям древних традиций, они указывают — пусть даже косвенно,
но все же реаль-

но — на знаки присутствия Бога в природе, что мы уже наблюдали прежде, когда
речь шла об учении, содержащемся в книгах Премудрости и посланиях апостола
Павла.

71. Будучи тесно связанными с людьми и с их историей, культуры подвержены той
же динамике, которая присутствует в истории человека. Следовательно, они
меняются и развиваются в результате контактов между людьми, которые
обмениваются различными моделями поведения. Культуры подпитываются тем, что
передают ценности, а их жизнеспособность и существование зависят от готовности
перенять новые элементы. Как объяснить эту динамику? Каждый человек вовлечен в
какую-то культуру, зависит от нее и воздействует на нее. Человек одновременно и
дитя, и отец культуры, в которой живет. Вся его жизнь наполнена тем, что
отличает его от остальных созданий: постоянной готовностью принять истину и
неутолимой жаждой знаний. В результате любая культура заключает в себе и
выражает стремление к некоему завершению. Можно сказать, что в самой культуре
кроется возможность принять Божественное Откровение.

То, как христиане выражают свою веру, тоже зависит от культуры их окружения. В
свою очередь, сам способ выражения веры является одним из факторов, постепенно
формирующих культуру. В каждую культуру христиане вносят неизменную истину
Бога, которую Он Сам явил в истории и культуре определенного народа. Таким
образом, в последующих столетиях заново происходит событие, свидетелями
которого были паломники, находившиеся в Иерусалиме в день Пятидесятницы. Слушая
апостолов, они спрашивали: «(...) Сии говорящие — не все ли галилеяне? Как же
мы слышим каждый собственное наречие, в котором родились, парфяне, и мидяне, и
еламиты, и жители Месопотамии, Иудеи и Каппадокии, Понта и Асии, Фригии и
Памфилии, Египта и частей Ливии, прилежащих к Киринее, и пришедшие из Рима,
иудеи и прозелиты, критяне и аравитяне, слышим их нашими языками говорящих о
великих делах Божиих?» (Деян 2, 7-11). Евангелие, проповедуемое в различных
культурах, требует веры от тех, кто его принимает, но не мешает им сохранять
культурное самосознание. Это не вызывает разделения, ибо отличительной
особенностью крещеных яв-

ляется всеобщность, которая может принять любую культуру, способствуя развитию
тех элементов, которые косвенно приводят ее к полному самовыражению в истине.

Из этого следует, что конкретная культура никогда не сможет стать критерием
оценки, а тем более конечным критерием истины по отношению к Божественному
Откровению. Евангелие не выступает против данной культуры, то есть при встрече
с ней не стремится лишить ее характерного содержания или навязать чуждые, не
соответствующие ей формы. Напротив, учение, проповедуемое христианином в миру и
в разнообразии культур, является подлинной формой освобождения от дисгармонии,
вызванной грехом, и в то же время призывом к обретению полноты истины. В
результате такой встречи с христианством культуры ничего не теряют, наоборот,
эта встреча побуждает их подготовиться к принятию евангельской истины, чтобы
они могли получить вдохновение для дальнейшего развития.

72. То, что евангелизация сначала встретила на своем пути греческую философию,
не означает, что не следует обращаться к другим школам мышления. В наши дни, по
мере того как Евангелие распространяется на территориях культур, которые до
этого находились за пределами сферы влияния христианства, перед инкультурацией
(проникновением веры в культуру) встают новые задачи. Наше поколение
сталкивается с проблемами, которые приходилось решать Церкви в первые века ее
существования.

Я имею в виду, прежде всего, страны Востока, в которых распространено много
древних религиозных и философских традиций. Среди них особое место занимает
Индия. С огромной духовной энергией индийская мысль ищет опыт, который,
освобождая человека от ограничений времени и пространства, обладает абсолютной
ценностью. В динамике этих стремлений к освобождению возникают основные
метафизические школы.

В наши дни христиане, особенно в Индии, должны выбрать из этого богатого
наследия элементы, которые соответствуют их вере, для обогащения христианской
мысли. В процессе изучения, вдохновленные соборной Декларацией Nostra aetate,
верующие должны руководствоваться определенными критериями. Первый —

это критерий универсализма человеческого духа, основные стремления которого
можно встретить в неизменной форме в самых различных культурах. Второй
критерий, вытекающий из первого, таков: когда Церковь впервые встречается с
великими культурами, она не может отказаться от наследия, полученного благодаря
приобщению к греко-латинской мысли. Отказавшись от этого наследия, она нарушила
бы спасительный замысел Бога, ведущего Свою Церковь во времени и истории.
Впрочем, этот критерий обязателен для Церкви любой эпохи, в том числе, для
Церкви будущего, которая обогатится тем, что получит благодаря нынешним
контактам с культурами Востока и из этого наследия почерпнет новые принципы,
чтобы начать плодотворный диалог с культурами, которые человечество сумеет
создать и развить в будущем. Наконец, в-третьих, не следует путать необходимую
заботу о сохранении специфики и оригинальности индийской философской мысли с
мнением, что данная культурная традиция с ее особенностями должна находиться в
изоляции и укрепляться в противостоянии другим традициям, что противоречило бы
самой природе человеческого духа.

Все, что я сказал об Индии, касается также наследия великих культур Китая,
Японии и других стран Азии, а также сокровищ традиционных культур Африки,
которые передаются в основном в рамках устного предания.

73. Как следует из этих рассуждений, надлежащую связь богословия с философией
следует рассматривать в двух аспектах: для богословия отправной точкой и
первоисточником всегда должно быть слово Божие, явленное в истории, а конечной
целью — не что иное, как постижение этого слова, постепенно углубляемое
очередными поколениями. С другой стороны, поскольку слово Божие является
Истиной (см. Ин 17, 17), его более глубокому пониманию, несомненно,
способствуют поиски истины человеком, т.е. философские размышления, которые
ведутся в соответствии с присущими им принципами. Дело не в том, чтобы в
богословских умозаключениях употребить то или иное понятие или фрагмент
выбранной философской системы; важно, чтобы разум верующего использовал
способность искать во время размышлений истину, в рамках некоего движения,
которое, беря начало из слова Божьего, стремилось

бы к лучшему пониманию этого слова. Разумеется, действуя между этими двумя
полюсами — словом Божиим и все более глубоким постижением его, — разум
находится как бы под охраной, в некоторой степени, им руководят, и благодаря
этому ему удается избежать путей, которые могли бы отдалить его от
богооткровенной истины, а в итоге — и от обычной истины; более того, ему
указываются пути, о существовании которых он и не подозревал. Благодаря такой
связи со словом Божиим, которое является отправной точкой и целью, философия
обогащается, ибо разум открывает новые и неожиданные горизонты.

74. Плодотворность такой связи подтверждает личный опыт великих христианских
богословов, которые зарекомендовали себя и как выдающиеся философы; созданные
ими ценные сочинения теоретического характера позволяют нам ставить их в один
ряд с выдающимися представителями античной философии. Это касается как Отцов
Церкви, среди которых следует упомянуть как св. Григория Богослова и св.
Августина, так и средневековых Учителей Церкви, в особенности, замечательную
триаду: св. Ансельма, св. Бонавентуру и св. Фому Аквинского. О плодотворной
связи философии со словом Божиим свидетельствуют смелые поиски более близких к
нам по времени мыслителей, среди которых я хотел бы упомянуть таких
представителей западного мира, как Д.Г. Ньюмен, А. Росмини, Жак Маритен, Этьен
Жильсон и Эдит Штейн, а из восточной культуры — ученых такого ранга, как В.С.
Соловьев, П.А. Флоренский, П.Я. Чаадаев, В.Н. Лосский. Разумеется, ссылаясь на
этих авторов, наряду с которыми можно также упомянуть и других, я не собираюсь
производить оценку их взглядов, а только представить их творчество как яркий
пример философских размышлений определенного типа, которые обогатились
благодаря взаимодействию с истинами веры. Несомненно одно: изучение пути
духовного развития этих мыслителей будет способствовать успешным поискам истины
и позволит с большей пользой использовать достигнутые результаты в служении
человеку. Хочется пожелать, чтобы эта великая философско-богословская традиция
нашла в наши дни и в будущем продолжателей и исследователей на благо Церкви и
человечества.

Различные типы философских размышлений

75. Представленная выше история связей веры с философией показывает, что
существуют различные типы философских размышлений с точки зрения их отношения к
христианской вере. Первыйтип — это философия, которая не зависит от
евангельского Откровения; она появилась в эпоху до рождения Искупителя,
позднееразвивалась в областях, еще не затронутых проповедью Евангелия.В этой
ситуации философия заявила о своем справедливом стремлении быть независимым
начинанием, то есть действовать по своим законам, опираясь на способности
разума. Несмотря на то чтомы сознаем пределы, полагаемые врожденной слабостью
человеческого разума, это стремление следует поддерживать и укреплять.Ибо
философские размышления, являясь поисками истины в естественной сфере, всегда —
по крайней мере, «имплицитно» — открыты для сверхъестественной реальности.

Более того, и в том случае, когда богословие использует понятия и аргументы
философии, следует сохранить требование правильно понимаемой автономности
мышления. Рассуждения, которые ведутся в соответствии со строгими рациональными
критериями, гарантируют получение результатов, которые везде будут признаны
истинными. И в этом случае подтверждается принцип, что благодать не упраздняет
природу, но совершенствует ее: согласие веры, объемлющее разум и волю, не
разрушает, но совершенствует способность самостоятельного мышления каждого
верующего человека, принимающего богооткровенные истины.

От этого требования, соответствующего природе человека, существенно отступает
теория так называемой «независимой» философии, которую исповедуют многие
современные философы. Она не провозглашает справедливую автономию философских
рассуждений, а скорее, требует признания самодостаточности мышления, что,
естественно, ничем не обосновано, ибо отказ от сокровищ истины, проистекающих
из Божественного Откровения, означает закрытие доступа к глубинному познанию
истины, что наносит вред самой философии.

76. Другим типом философских размышлений является философия, которую многие
называют христианской. Этот термин сам

по себе обоснован, но его не следует понимать превратно: он не должен наводить
на мысль, что существует какая-то официальная философия Церкви, ибо вера как
таковая не является философией. Скорее, он обозначает христианские философские
размышления, умозрительные рассуждения, которые возникли в живой связи с верой.
Следовательно, здесь не имеется в виду только философия, созданная
христианскими философами, которые в своих поисках не хотели вступать в
противоречие с верой. Говоря о христианской философии, мы имеем в виду все
важнейшие направления философской мысли, которые не возникли бы без прямого или
косвенного вклада христианской веры.

Итак, можно говорить о двух аспектах христианской философии, из которых один
является субъективным и заключается в очищении разума верой. Являясь
богословской добродетелью, вера освобождает разум от самонадеянности —
типичного искушения, которому легко поддаются философы. Ее осуждали еще апостол
Павел и Отцы Церкви, а в более близкое к нам время — такие философы, как
Паскаль и Кьеркегор. Благодаря смирению, философ находит в себе смелость
рассматривать проблемы, которые ему было бы трудно решить, если бы он не
учитывал знаний, полученных из Откровения. Примерами могут служить: проблема
зла и страдания, персонификация Бога, вопрос о смысле жизни, а более
непосредственно — радикальный метафизический вопрос: «Почему нечто
существует?».

Есть также объективный аспект, касающийся содержания философии: Откровение
являет определенные истины, которые разум, быть может, никогда бы не открыл,
если бы рассчитывал только на собственные силы, хотя они и доступны его
естественному познанию. К этой сфере относится понятие личностного и свободного
Бога-Творца, которое сыграло столь значительную роль в развитии философской
мысли, и прежде всего, философии бытия. Сюда также относится реальность греха,
так, как она представлена в свете веры, которая помогает правильно описать в
философских категориях проблему зла. Концепция личности как духовной сущности
тоже является особенной заслугой веры; не будем забывать и о том, что
христианское учение о достоинстве, равенстве и свобо-

де людей, несомненно, повлияло на современные философские размышления. В более
близкое нам время можно отметить открытие значимости для философии того
исторического события, которое является кульминацией христианского Откровения.
Не случайно в истории философии оно рассматривается как ключевой факт, который
открывает новую главу в поисках истины человеком.

Объективным элементом христианской философии является также необходимость
изучить рациональность некоторых истин Священного Писания, таких как
возможность сверхъестественного призвания человека или первородный грех. Эти
начинания вынуждают разум признать, что существует истина и рациональная
действительность далеко за пределами тех узких рамок, которыми он был склонен
себя ограничить. Проблематика такого рода существенно расширяет область
рациональных размышлений.

Рассматривая эти вопросы, философы не стали богословами, ибо не стремились к
пониманию и объяснению истин веры в свете Откровения. Они продолжали работать в
своей области, используя собственные, чисто рациональные методы, но
одновременно расширяли территорию своих поисков за счет новых областей истины.
Можно сказать, что без стимулирующего влияния слова Божьего не появилась бы
большая часть философии нового времени и XX века. Этот факт сохраняет свое
значение даже несмотря на то, что многие мыслители последних столетий, к
сожалению, отошли от христианской ортодоксии.

77. С другим важным типом философских размышлений мы сталкиваемся тогда, когда
само богословие обращается к философии. В действительности богословие всегда
нуждалось и нуждается до сих пор в помощи философии. Поскольку богословие
создано критическим разумом, направляемым светом веры, основой и непременным
условием всех его исканий является разум, надлежащим образом ознакомленный с
философскими понятиями и аргументами. Кроме того, философия необходима
богословию, чтобы вести с ней диалог с целью выяснения, являются ли
богословские утверждения понятными и универсально истинными. Не случайно Отцы
Церкви и средневековые богословы использовали нехристианские философии, чтобы
доверить им разъяснительную функцию.

Этот исторический факт подчеркивает значение автономности, которую сохраняет
также этот третий тип философских размышлений, но одновременно указывает на
необходимые и глубокие преобразования, которым философия должна подвергнуться.

Именно по причине этой необходимой и значительной помощи уже в эпоху Отцов
Церкви философию называли служанкой богословия (ancilla theologiae). Это
определение не означало, что философия безвольно подчиняется богословию или
играет по отношению к нему чисто функциональную роль. Скорее всего, его
понимали в том смысле, в каком Аристотель называл экспериментальные науки
«служанками первой философии». Хотя в наши дни было бы трудно использовать это
определение из-за упомянутого выше принципа автономии, его успешно использовали
на протяжении веков, чтобы указать на необходимость связи между этими двумя
областями знаний и на невозможность их разделения.

Если богослов не хочет использовать философию, возникает опасность, что он
будет бессознательно создавать собственную философию и замкнется в структурах
мышления, которые мало подходят для постижения веры. Если бы философ, со своей
стороны, отказался от всех связей с богословием, он вынужден был бы
постулировать истины христианской веры как самостоятельные принципы, что и
приходилось делать некоторым философам нового времени. В обоих случаях
возникает опасность уничтожения основ автономии, которую справедливо пытается
сохранить любая наука.

Рассматриваемая здесь разновидность философии, в связи с ее влиянием на
понимание Откровения, вместе с богословием должна строго подчиняться суждениям
и авторитету Учительства Церкви, как было показано выше. Истины веры выдвигают
определенные требования, которые философия должна соблюдать, если желает
поддерживать связь с богословием.

78. В свете этих рассуждений легко понять, почему Учительство Церкви
неоднократно превозносило достоинства идей св. Фомы и указывало на него как на
лучшего, образцового богослова. Оно не хотело таким образом выражать свое
отношение к чисто философским вопросам или требовать принятия определенных
взглядов. Цель Учительства по-прежнему заключается в том, чтобы предста-

вить св. Фому как образец для всех людей, ищущих истину. В его рассуждениях
требования разума и сила веры соединились в наиболее возвышенном синтезе,
который когда-либо создавала человеческая мысль, ибо он умел решительно
защищать новые истины, явленные в Откровении, никогда не нарушая принципов,
которыми руководствуется разум.

79. Развивая положения, содержащиеся в предшествующих высказываниях Учительства
Церкви, я намерен в последней части настоящей энциклики указать некоторые
требования, которые богословие и, прежде всего, слово Божие, предъявляют в наши
дни философской мысли и современным философам. Как я уже подчеркивал, философ
должен следовать собственным правилам и опираться на свои принципы, но истина
может быть только одна. Истины Откровения не могут приуменьшать открытий разума
и его справедливой автономии; но, в свою очередь, разум никогда не должен
терять способность размышлять о себе самом и задавать вопросы, сознавая, что он
не может придать себе абсолютный и исключительный статус. Богооткровенная
истина, являя тайну бытия во всем свете, который исходит от Самого Творца,
дающего жизнь, должна осветить путь философских размышлений. Следовательно,
христианское Откровение становится связующим элементом и местом встречи
философской и богословской мысли в их взаимной соотнесенности. Хочется
пожелать, чтобы богословы и философы подчинялись лишь авторитету истины, чтобы
создать философию, созвучную слову Божьему. Такая философия станет местом
встречи культуры и христианской веры, основой для взаимопонимания верующих и
неверующих. Она сможет помочь верующим осознать глубину и подлинность веры,
когда ей сопутствуют размышления и она не отказывается от них. Это убеждение
подтверждают также Отцы Церкви: «Вера сама по себе является ничем иным, как
согласием разума. (...) Размышляет всякий, кто верует — и, веруя, размышляет,
и, размышляя, верует. (...) Если вера не является предметом размышлений, то ее
вовсе не существует»95 . А также: «Если нет согласия разума, то нет и веры, ибо
без согласия разума невозможно во что-либо верить»96 .

\section{Нужды и задачи настоящего времени}

Неотъемлемые требования слова Божьего

80. В Священном Писании можно обнаружить целый ряд элементов — доступных
непосредственно или косвенно, — которые позволяют сформировать представление о
человеке и мире, обладающее значительной философской ценностью. Христиане
постепенно осознали, какое богатство заключено в этих святых страницах. В
Библии показано, что действительность, которую мы воспринимаем, — не абсолют:
она не является несотворенной и не родилась сама по себе. Только Бог — Абсолют.
На страницах Священного Писания появляется также концепция человека как образа
Божия, в которой заключены достоверные истины, касающиеся его сущности и
свободы, а также бессмертия души. Поскольку сотворенный мир не является
самодостаточным, любое обманчивое чувство автономии, игнорирующее основную
зависимость каждого творения, в том числе человека, от Бога, приводит к
трагедиям, которые губят рациональные поиски гармонии и смысла человеческого
существования.

Проблема нравственного зла, наиболее трагическая форма проявления зла, тоже
рассматривается в Библии, где говорится, что его причины не кроются в каких-то
недостатках, связанных с материей, но что зло есть рана, которая причиняется
человеком, неупорядоченно пользующимся свободой. Наконец, слово Божие ставит
вопрос о смысле бытия и отвечает на него, направляя человека к Иисусу Христу,
Воплощенному Слову Божьему, Которое во всей полноте осуществляет замысел
человеческой жизни. Изучение священного текста позволит заметить и другие
аспекты, но на первом плане выступает отказ от любых форм релятивизма,
материализма и пантеизма.

Фундаментом этой философии, заключенной в Библии, является убеждение, что
человеческая жизнь и мир имеют смысл и стремятся к полноте, которая
осуществляется в Иисусе Христе. Вопло-

щение всегда будет оставаться главной тайной, к которой следует обращаться,
чтобы постичь загадку существования человека, сотворенного мира и самого Бога.
Эта тайна ставит перед философией самое суровое требование, ибо велит разуму
усвоить логику, которая позволит разрушить стены, в которых он может сам себя
заточить. Однако лишь на этом пути смысл человеческой жизни достигает своей
вершины, ибо на нем становится понятной глубочайшая сущность Бога и человека: в
тайне Воплощенного Слова сохраняется и Божественная, и человеческая природа, а
также автономия каждой из них, и в то же время возникает отношение любви,
которое соединяет их друг с другом, а также единственная в своем роде связь,
которая, избегая смешения двух естеств, выражает их во взаимном единении97 .

81. Можно заметить, что одним из наиболее характерных аспектов нашего нынешнего
положения является «кризис смысла». Появилось так много концепций познания,
часто научного характера, отражающих видение жизни и мира, что на самом деле
все шире распространяется так называемая фрагментация знаний. Именно это
приводит к тому, что поиски смысла затруднены, а часто — и бесплодны. Более
того, и это еще печальнее, в обстановке переплетения информации и фактов, среди
которых мы живем и которые, похоже, становятся содержанием нашей жизни, многие
сомневаются, уместен ли еще сам вопрос о смысле жизни. Многообразие теорий,
соревнующихся в попытках решить этот вопрос, а также многочисленные концепции и
интерпретации мира и человеческой жизни только усиливают эти сомнения, которые
легко могут стать источником скептицизма и равнодушия или различных видов
нигилизма.

В результате человеческим духом овладевает какой-то неопределенный вид
мышления, который приводит к еще большей замкнутости в границах собственной
имманентности, без всяких обращений к трансцендентному. Философия, которая не
ставит вопросов о смысле бытия, несет серьезную опасность, которая заключается
в отведении разуму чисто инструментальных функций и потере подлинного интереса
к поискам истины.

Чтобы быть созвучной слову Божьему, философия должна,

прежде всего, обрести глубину мудрости для поисков окончательного и абсолютного
смысла жизни. Это основное условие, в сущности, является необходимым стимулом
для философии, чтобы она приспособилась к собственной природе, ибо, следуя этим
путем, философия не только станет решающим критическим авторитетом, который
указывает различным научным дисциплинам их основы и ограничения, но и конечной
инстанцией, объединяющей человеческие знания и деятельность, благодаря тому,
что под ее влиянием они будут стремиться к единой высшей цели и смыслу. Эта
глубина мудрости сегодня особенно необходима, так как колоссальный рост
технического потенциала заставляет человечества заново и со всей остротой
осмыслить высшие ценности. Если эти технические средства не будут подчинены
какой-то цели, которая выходит за рамки логики чистого утилитаризма, они вскоре
могут показать свой античеловеческий характер, и даже превратиться в
потенциальные средства уничтожения человеческого рода98 .

Слово Божие являет конечную цель человека и придает полноценный смысл его
действиям в мире. Именно поэтому оно призывает философию найти для этого смысла
естественный фундамент, которым является религиозность, заключенная в природе
каждого человека. Философия, которая подвергает сомнению существование
совершенного и абсолютного смысла, является не только непригодной, но и ложной.

82. Кроме того, эту роль мудрости может играть лишь философия, являющая собой
подлинную и истинную науку, то есть имеющая своим предметом не только частные и
преходящие аспекты реальности — функциональные, формальные и практические, но
ее полную и окончательную истину, т.е. саму сущность объектов познания. Итак,
мы подходим ко второму постулату: следует показать, что человек способен прийти
к познанию истины; причем имеется в виду познание, которое находит объективную
истину в соответствии вещи и интеллекта, как определяли ее великие схоласты9 9.
Это требование, свойственное вере, было безоговорочно подтверждено на Втором
Ватиканском Соборе: «Ведь разум не сводится к познанию одних лишь внешних
явлений: он способен с под-

линной достоверностью постигать доступную ему реальность, несмотря даже на то,
что вследствие греха он частично помрачается и ослабляется»100 .

Философия, которой свойствен феноменализм или релятивизм, не способна помочь
постичь сокровища, заключенные в слове Божием, ибо Священное Писание
предполагает, что, хотя человек и запятнал себя недобросовестностью и ложью, он
может познать ясную и простую истину. В Священном Писании, особенно в Новом
Завете, встречаются тексты и утверждения, носящие подлинно онтологический
характер. Богодухновенные авторы хотели дать в них истинные суждения, то есть
такие, которые отражают объективную действительность. Нельзя утверждать, что
католическая традиция совершила ошибку, истолковав некоторые тексты апостола
Иоанна Богослова и святого апостола Павла как утверждения, касающиеся самой
сущности Христа. Следовательно, когда богословие пытается постичь и объяснить
эти утверждения, оно нуждается в помощи философии, которая не исключает
возможности объективно истинного познания, даже если его можно продолжать
совершенствовать. Все, что здесь было сказано, касается также суждений совести,
причем в Священном Писании считается, что они могут быть объективно
истинными101 .

83. Из двух приведенных выше требований следует третье: необходима философия
истинно метафизического характера, т.е. способная выйти за рамки данных,
полученных опытным путем, чтобы в поисках истины открыть нечто абсолютное,
конечное и фундаментальное. Это требование касается как интеллектуального, так
и аналитического познания; особенно оно типично для познания нравственного
блага, конечной основой которого является высшее Благо, сам Бог. Я не хочу
говорить здесь о метафизике как о конкретной школе и течении, возникшем в
прошлом. Я лишь хочу отметить, что действительность и истина выходят за пределы
фактов и опыта; хочу также выступить в защиту способности человека достоверно и
безошибочно, хотя и несовершенными методами, с помощью аналогий, постигать это
трансцендентное и метафизическое измерение. В таком понимании метафизику не
следует считать альтернативой антропологии, так как именно метафизика позволяет

обосновать понятие достоинства личности, указывая на ее духовную природу.
Проблематика личности является особенно благоприятной сферой, в которой
происходит встреча с бытием, а тем самым — с метафизическими рассуждениями.

Везде, где человек замечает призыв к абсолютному и трансцендентному, перед ним
открывается путь к метафизическому аспекту действительности: в истине, в
красоте, в нравственных ценностях, в другом человеке, в тварном бытии, в Боге.
Важнейшая задача, которая поставлена перед нами в конце нынешнего тысячелетия,
причем неотложная и неизбежная, заключается в переходе от явления к основанию.
Нельзя довольствоваться только опытом; и в том случае, когда он выражает и
раскрывает внутренний мир человека и его духовность, рациональные размышления
тоже должны восходить к духовной субстанции и к тому основанию, на котором она
зиждется. Следовательно, философские размышления, которые отказываются от
принятия метафизики, совершенно непригодны для выполнения посреднической
функции в постижении Откровения.

Слово Божие постоянно обращается к тому, что находится вне опыта и даже вне
человеческой мысли; но эту «тайну» не удалось бы раскрыть, а богословие не
смогло бы сделать ее в какой-то мере понятной102 , если бы человеческое
познание было ограничено лишь миром, воспринимаемым органами чувств. Поэтому
метафизика играет особенно важную роль в богословских исследованиях.
Богословие, лишенное метафизического измерения, не смогло бы выйти за рамки
анализа религиозного опыта, а осознание веры не смогло бы точно выразить
универсальную и трансцендентную ценность богооткровенной истины.

Я придаю такое значение метафизике, поскольку убежден, что только таким путем
можно преодолеть кризис, который в настоящее время затронул обширные области
философии, и исправить некоторые неправильные установки, распространенные в
нашем обществе.

84. Необходимость авторитета метафизики станет более очевидной, если мы обратим
внимание на развитие герменевтических наук и различных форм языкового анализа в
настоящее время. Результаты этих поисков могут оказаться полезными для
понимания ве-

ры, так как они раскрывают структуру нашего мышления и речи, а также смысл,
заключенный в языке. Однако отдельные представители этих наук сводят поиски к
одному вопросу, а именно, каким образом человек постигает и отображает
действительность, и не пытаются установить, в состоянии ли разум раскрыть ее
сущность. Разве такое отношение не является еще одним проявлением современного
кризиса веры в способности разума? Если же, основываясь на каких-то
произвольных положениях, вышеупомянутые суждения пытаются затмить истины веры
или подорвать их универсальную истинность, то они не только умаляют роль
разума, но и вступают в противоречие друг с другом, ибо вера предполагает, что
человеческим языком можно универсально выразить, хотя и в аналогичных, но не
менее возвышенных категориях, Божественную и трансцендентную реальность103 .
Если это неверно, то слово Божие, которое является словом Бога, изложенным на
языке человека, не могло бы передать знаний о Боге. Нельзя, чтобы в процессе
постижения этого слова постоянно возникали очередные интерпретации, что не
позволяло нам прийти к истинному суждению; в противном случае Божественное
Откровение было бы невозможно, имелись бы лишь чисто человеческие представления
о Боге и о том, что мы считаем Его планами относительно нас.

85. Я отдаю себе отчет в том, что те требования, которые слово Божие
предъявляет к философии, могут показаться трудными для многих людей,
непосредственно занимающихся современными философскими исследованиями. И все
же, основываясь на систематическом учении Римских Пап последнего времени и
Отцов Второго Ватиканского Собора, я хочу выразить мое глубокое убеждение, что
человек способен выработать однородную и органичную концепцию познания. Это
одна из задач, которая поставлена перед христианской мыслью в третьем
тысячелетии эры христианства. Раздробленность знаний связана с неполным
видением истины и приводит к фрагментации смысла, а это не позволяет
современному человеку достичь внутреннего единства. Разве Церковь может
остаться к этому равнодушной? Само Евангелие ставит перед пастырями задачу
служения мудрости, и поэтому они не могут уклониться от этой обязанности.

Я убежден, что философы, которые в настоящее время пытаются решить задачи,
поставленные перед человеческой мыслью словом Божиим, должны опираться в своих
размышлениях на вышеизложенные постулаты и регулярно обращаться к великой
традиции, начатой древними философами, продолженной Отцами Церкви, а также
знаменитыми схоластами и включающей фундаментальные достижения нового времени и
XX века. Философ, который умеет черпать из этой традиции и находить в ней
вдохновение, сможет сохранить автономию философской мысли.

Поэтому в нынешней ситуации особенно важно, чтобы отдельные философы смогли
пропагандировать заново открытую решающую роль традиции в формировании
надлежащей формы познания. Ибо следование традиции — это не только воспоминание
о прошлом; в большей степени оно выражает признание ценности культурного
наследия, принадлежащего всем людям. Можно даже сказать, что это мы принадлежим
традиции и не в праве ею самовольно распоряжаться. Именно связь с традицией
позволит нам сформулировать оригинальную и новую мысль, открывающую перспективы
на будущее. В большей степени принцип связи с традицией касается богословия, не
только потому, что его первоисточником является живое Предание Церкви104, но
также потому, что именно этот факт требует заново принять как глубокую
богословскую традицию, сыгравшую столь важную роль в минувших эпохах, так и
непреходящую традицию философии, которая благодаря своей истинной мудрости
смогла преодолеть границы времени и пространства.

86. Внимание, которое уделяется необходимости сохранить непрерывную связь
современных философских размышлений с идеями, появившимися в христианской
традиции, должно устранить опасность, кроющуюся в некоторых течениях мысли,
широко распространенных в наши дни. Я считаю, что следует описать их хотя бы
вкратце, чтобы обратить внимание на имеющиеся в них ошибки, представляющие
угрозу для философии.

Первое из этих течений получило название эклектизма. Этот термин определяет
позицию того, кто в своих исканиях, учении и аргументации — в том числе в сфере
богословия — имеет при-

вычку использовать отдельные мысли из различных философских концепций, невзирая
ни на их целостность и систематическую связь, ни на исторический контекст.
Таким образом, он лишает себя возможности отделить крупицу истины, заключенной
в данной мысли, от ложного и ошибочного. Крайней степенью эклектизма можно
считать использование философских терминов в чисто риторических целях, чем
иногда грешат некоторые богословы. Подобные злоупотребления не облегчают поиск
истины и не формируют ни богословского, ни философского навыка мышления,
основанного на серьезной научной аргументации. Систематическое и глубокое
изучение философских учений, характерного языка и обстоятельств, при которых
они появились, поможет избежать опасности эклектизма и позволит использовать их
надлежащим образом в богословской аргументации.

87. Эклектизм является методологической ошибкой, но может также содержать
скрытые тезисы историзма. Чтобы правильно понять учение, возникшее в прошлом,
необходимо поместить его в должный исторический и культурный контекст.
Характерная же особенность историзма заключается в том, что он считает некую
философию истинной, если она отвечает требованиям данной эпохи и выполняет
поставленные перед нею исторические задачи. Таким образом, по крайней мере,
косвенно, ставится под сомнение неизменная ценность истины. То, что истинно в
одной эпохе, утверждают сторонники историзма, может не быть истинным в другой.
В результате история мысли превращается для них в хранилище древностей, из
которого можно черпать примеры взглядов, которые ныне в значительной степени
устарели и потому не имеют никакого значение для современной эпохи. В
действительности, однако, следует помнить, что даже если сама формулировка
истины каким-то образом подчинена требованиям определенной эпохи и культуры,
все же в любых суждениях можно выявить истину или заблуждение и соответственно
признать их истинными или ложными независимо от удаленности во времени и
пространстве.

В богословской мысли историзм обычно проявляется в виде своеобразного
«модернизма». Под влиянием в какой-то мере правильной заботы об актуальности
богословских размышлений и их

доступности для современного человека богословы склонны пользоваться
исключительно новейшими формулировками, взятыми из философского жаргона, и уже
не относятся к ним с должной критикой, которая необходима в свете традиции.
Такой вид модернизма ошибочно отождествляет актуальность с истиной и поэтому не
может выполнить требований истины, которым должно отвечать богословие.

88. Очередной опасностью, которую следует рассмотреть, является сциентизм. Эта
философская концепция не признает ценности иных видов познания, кроме тех,
которые свойственны точным наукам, считая плодом воображения как религиозное и
богословское познание, так и знания в области этики и эстетики. В прошлом ту же
идею провозглашали позитивизм и неопозитивизм, согласно которым утверждения
метафизического характера лишены смысла. Эпистемологическая критика показала
необоснованность этой теории, но она возродилась в виде сциентизма, в котором
ценности сводятся к уровню обычных ощущений, а понятие бытия отодвигается на
второй план, так как рассмотрению подлежит лишь то, что относится к области
«голых» фактов. Таким образом, наука, пользуясь развитием техники, готовится
подчинить себе все аспекты жизни человека. Бесспорные достижения науки и
современной техники способствуют распространению сциентистского мышления.
Создается впечатление, что его воздействие ничем не ограничено, так как оно
проникло в различные культуры, вызвав в них радикальные перемены.

К сожалению, все, что касается вопроса о смысле жизни, сциентизм относит к
иррациональной сфере или к воображению. Трудно также согласиться с отношением
этого направления к другим важнейшим вопросам философии, которые оно или
полностью игнорирует, или анализирует с помощью поверхностных аналогий,
лишенных рациональных обоснований. Это приводит к оскудению размышлений
человека в результате устранения из их сферы тех фундаментальных вопросов,
которые человек как разумное существо с самого начала своей жизни на земле
постоянно задавал себе. Кроме того, согласно этому учению, недопустимы любые
критические высказывания, основанные на этической оценке, и поэтому сциен-

тистское мышление смогло внушить многим людям, что все технически осуществимое
допустимо с точки зрения нравственности.

89. Не меньшую опасность представляет прагматизм — образмышления тех, кто,
совершая выбор, не видит необходимости обращаться к теоретическим рассуждениям
и к оценкам, основанным на этических принципах. Это направление мысли имелона
практике важные последствия. Прежде всего, оно привело к появлению концепции
демократии, в которой нет места любым упоминаниям о принципах аксиологического
характера (а следовательно, неизменных); о допустимости или недопустимости
определенного поведения решает парламентское большинство105 . Последствия такой
концепции очевидны: важнейший нравственный выбор человека в действительности
зависит от сиюминутных решений каких-то учреждений. Более того, даже
антропология в большой степени обедняется из-за характерной для прагматизма
односторонней концепции человека, не допускающей ни сложных этических дилемм,
ни экзистенциальных размышлений о смысле страдания и жертвы, жизни и смерти.

90. Взгляды, представленные выше, ведут, в свою очередь, к более общей
концепции, которая, по-видимому, определяет сегодняточку зрения многих
философских направлений, утративших интерес к вопросам бытия. Я имею в виду
нигилистическое отношение, которое отрицает одновременно все основы и любую
объективную истину. Нигилизм не только противоречит требованиями истинам слова
Божьего, но, прежде всего, отрицает гуманностьчеловека и его неотъемлемые
свойства. Нельзя забывать, что недостаточное внимание к вопросам бытия
неуклонно приводит к потере контакта с объективной истиной, а в результате — с
основой,на которой покоится человеческое достоинство. Таким образом, появляется
возможность лишить облик человека тех черт, которыесвидетельствуют о его
сходстве с Богом, чтобы постепенно пробудить в нем деструктивную жажду власти
или погрузить его в пучину одиночества, порождающего отчаяние. Если человека
лишитьистины, все попытки его освобождения становятся неосуществимыми, ибо
истина и свобода или существуют вместе, или же вместе жалким образом
погибают106 .

91. Комментируя перечисленные выше направления мысли, я не ставил перед собой
задачу дать полную картину современной философии; впрочем, было бы трудно
представить ее в рамках какой-то единой концепции. Я должен подчеркнуть, что в
действительности многие науки пополнились сокровищами знаний и мудрости.
Достаточно упомянуть логику, философию языка, эпистемологию, философию природы,
антропологию, углубленный анализ чувственного восприятия, экзистенциальные
концепции свободы. С другой стороны, получивший широкое распространение
имманентизм, являющийся центральным элементом рационалистических притязаний,
уже вызвал в прошлом веке ряд действий, которые привели к радикальному
отрицанию постулатов, считавшихся нерушимыми. Так появились иррациональные
течения, а критический анализ показал тщетность стремлений разума к полному
самообоснованию.

Некоторые мыслители называют нашу эпоху «эпохой постмодернизма». Этот термин,
часто используемый в различных контекстах, указывает на появление целого ряда
новых факторов, воздействие которых было столь сильным и широким, что привело к
существенным и стойким изменениям. Первоначально это определение применялось по
отношению к явлениям эстетического, общественного или технологического
характера. Позже оно стало использоваться в философии, но сохранило некоторую
двусмысленность, как в связи с тем, что термин «постмодернизм» иногда носит
положительный, а иногда отрицательный характер, так и потому, что не существует
общепринятого решения трудной проблемы проведения границы между двумя
историческими эпохами. Одно не подлежит сомнению: течения мысли, ссылающиеся на
постмодернизм, заслуживают внимания. В некоторых из них утверждается, что эпоха
достоверных суждений безвозвратно прошла и теперь человек должен научиться жить
в ситуации полного отсутствия смысла, под знаком бренности и изменчивости
вещей. Многие авторы, отрицая всякую достоверность в философии, забывают о
необходимом разграничении и доходят до отрицания достоверности самих истин
веры.

Этот нигилизм находит своего рода подтверждение в страшном

опыте зла, являющемся характерной чертой нашей эпохи. Из-за этого трагического
опыта потерпел крушение рациональный оптимизм, который воспринимал историю как
победное шествие разума, источник счастья и свободы; в результате, одной из
самых серьезных опасностей на сегодняшний день, в конце века, является
искушение отчаяния.

Однако до сих пор параллельно существует и своего рода позитивистское мышление,
которое не избавилось от заблуждения, что благодаря достижениям науки и техники
человек может, словно демиург, обеспечить полный контроль над своей судьбой.

Актуальные задачи богословия

92. В различных эпохах перед богословием, представляющим собой не что иное, как
истолкование Откровения, основанное на умозаключениях, всегда стояла задача
принять наследие различных культур, а затем приобщить их к истинам веры с
помощью понятий, соответствующих этим культурам. В наши дни богословию тоже
вверена двоякая миссия. С одной стороны, оно должно выполнить задачу, которую в
свое время поставили перед ним Отцы Второго Ватиканского Собора, то есть
обновить методологию, чтобы более эффективно служить делу евангелизации. В этом
контексте нельзя не привести слова Папы Иоанна XXIII, которые он произнес на
открытии Собора: «В соответствии с горячим желанием всех тех, кто искренне
возлюбил христианскую, католическую и апостольскую веру, эта наука должна более
широко изучаться и глубже постигаться, а человеческому разуму следует
обеспечить более полное образование и воспитание в этой области; необходимо,
чтобы безошибочное и неизменное умение, которому следует хранить верность и
оказывать уважение, углублялось и представлялось в соответствии с требованиями
нашего времени»107 .

С другой стороны, богословие должно устремлять взор к конечной истине,
переданной через Откровение, не задерживаясь на промежуточных этапах. Богослов
обязан помнить, что в его труде выражается «динамика, присущая вере», а
подлинным объектом его поисков является «истина, живой Бог и Его план спасения,
яв-

ленный в Иисусе Христе»108 . Хотя эта задача и касается в первую очередь
богословия, она является также вызовом философии, ибо масса проблем, с которыми
обе эти науки сегодня сталкиваются, призывает их к сотрудничеству, с
сохранением, однако, собственных методологий, чтобы истина могла быть заново
осознана и выражена. Истина, которой является Христос, требует принять ее как
универсальный авторитет, который является основой как богословия, так и
философии, а также стимулирует их и позволяет им возрастать (см. Еф 4, 15).

Вера в возможность познания истины, имеющей универсальную ценность, отнюдь не
вызывает нетерпимость; напротив, она является необходимым условием искреннего и
подлинного диалога между людьми. Только выполнив это условие, люди смогут
преодолеть разделение и вместе стремиться к познанию полноты истины, следуя
путями, которые известны лишь Духу воскресшего Господа109 . Сейчас я хочу
показать, в каких конкретных формах, с точки зрения актуальных задач
богословия, выражается в наши дни это требование единства.

93. Основной целью богословия является постижение Откровения и истин веры.
Следовательно, подлинным средоточием богословских размышлений должно быть
созерцание тайны Триединого Бога. Путь к ней открывают рассуждения о тайне
ВоплощенияСына Божьего, то есть о том, что Он стал человеком и в
результатепринял страдания и смерть, а затем воскрес во славе, восшел на небеса
и воссел одесную Отца, откуда прислал Духа Истины, чтобыОн установил Церковь и
оживотворял ее. Первостепенной задачейбогословия в этом контексте является
постижение уничижения Бога, которое действительно является великой тайной для
человеческого разума, ибо он не может понять, что страдание и смерть могут
выражать любовь, которая приносит себя в дар, не требуя ничего взамен. С этой
точки зрения насущной необходимостью становится внимательный анализ текстов,
прежде всего библейских, а затем и тех, в которых выражено живое предание
Церкви. В связис этим в наше время возникает немало вопросов, отчасти и
новых,которые невозможно правильно решить без участия философии.

94. Первый вопрос касается соотношения смысла и истины.

В текстах, которые интерпретирует богослов, как и в любом другом тексте,
передан, прежде всего, некоторый смысл, который следует найти и выразить. В
данном случае смысл представляет собой истину о Боге, переданную самим Богом
через священный текст. Человеческий язык становится воплощением языка Бога,
Который являет Свою истину, таинственным образом «снисходя» до нашего уровня, в
соответствии с логикой Боговоплощения110 . Поэтому богослов, интерпретирующий
источники Откровения, должен задуматься, какую глубокую и неискаженную истину
содержат тексты, независимо от ограничений, которые навязывает им язык.

Несомненно, истина библейских текстов, особенно Евангелия, заключается не
только в том, что они описывают обычные исторические события или нейтральные
факты, как того хочет исторический позитивизм111 . Наоборот, в этих текстах
говорится о фактах, истинность которых не следует исключительно из их
исторического характера, а заключена в их значении в истории спасения и для
нее. Эта истина полностью раскрывается Церковью, которая на протяжении веков
продолжает толковать эти тексты, сохраняя в неприкосновенности их смысл. Итак,
существует неотложная необходимость, чтобы и с позиций философии поставить
вопрос о соотношении факта и его значения — о связи, которая придает
специфический смысл истории.

95. Слово Божие не предназначено только для одного народа или одной эпохи.
Догматические постановления также выражают постоянную и окончательную истину,
хотя в них иногда заметно влияние культуры той эпохи, в которой они были
сформулированы. Приходит в голову очевидный вопрос: как можно совместить
абсолютный и универсальный характер истины с тем фактом, что формулировки, в
которых она выражена, неизбежно зависят от исторических и культурных факторов.
Как я уже сказал ранее, тезисы историзма не выдерживают критики. А благодаря
использованию герменевтики, принимающей метафизическое измерение, можно
показать, каким образом совершается переход от исторических и внешних
обстоятельств, в которых создавались данные тексты, к выраженной в них истине,
выходящей за рамки этих ограничений.

Человек может своим ограниченным и развивавшимся в ходе истории языком выразить
истины, которые выходят за пределы языка, ибо истина не может быть ограничена
рамками времени и культуры; она постигается в истории, но сама по себе выше
истории.

96. Это мышление позволяет решить еще одну проблему,а именно, проблему
неизменной ценности понятий, используемыхв определениях Соборов. Уже мой
досточтимый предшественникПий XII рассматривал эту проблему в энциклике Humani
generis112 .

Нелегко решить эту проблему, так как необходимо внимательно анализировать
значения, которые отдельные слова имеют в различных культурах и эпохах. Тем не
менее, история мысли показывает, что некоторые основные понятия сохраняют
универсальную познавательную ценность в различных культурах и на очередных
этапах их эволюции, таким образом, сохраняются истинные формулировки, которые
их выражают113 . Если бы было иначе, философия и различные научные дисциплины
не смогли бы найти общий язык и не воспринимались бы культурами, отличными от
тех, в которых они родились и развивались. Итак, проблема герменевтики
существует, но ее можно решить. С другой стороны, объективная реальность,
скрывающаяся за многими понятиями, часто сочетается с неясностью самих понятий,
и здесь философские размышления могли бы оказать ощутимую помощь. Желательно,
чтобы более глубоко изучались связи между языком понятий и истиной, а также
были указаны подходящие пути, ведущие к правильному пониманию этих связей.

97. Хотя важной миссией богословия является интерпретацияисточников, ее
очередной, при этом более сложной и ответственной задачей является понимание
богооткровенной истины, то естьпроцесс осознания веры. Как я уже упоминал,
осознание веры нуждается в поддержке философии бытия, которая позволяет,
преждевсего, догматическому богословию, должным образом выполнятьсвои функции.
Появившийся в начале века догматический прагматизм, согласно которому истины
веры являются не более чем нравственными нормами, уже был подвергнут критике и
отвергнут114 ;несмотря на это, некоторые до сих пор испытывают искушение
понимать эти истины только функционально. Однако это приводит

к принятию неподходящей схемы, ограниченной и «редуктивистской». Например, так
называемая «христология, идущая снизу», или экклезиология, которая использует в
качестве образца только модели гражданских обществ, не смогли бы избежать
опасности такого редукционизма.

Если осознание веры должно объять все богатство богословской традиции, ему
необходимо обращаться к философии бытия. Такая философия должна суметь заново
сформулировать проблему бытия в соответствии с требованиями и наследием всей
философской традиции, включая и философию нового времени, избегая бесплодного
тиражирования устаревших схем. Философия бытия в рамках христианской
метафизической традиции является динамичной философией, воспринимающей
действительность в онтологических, причинных и коммуникативных структурах. Она
черпает силу и постоянство из того, что ее отправной точкой является сам акт
существования, что позволяет ей полностью принять всю действительность,
преодолевать все границы и даже достигать Того, в Ком все вещи обретают свое
завершение115 . В богословии, которое берет начало в Откровении как в новом
источнике познания, принятие этой концепции обосновано тесной связью веры с
метафизической рациональностью.

98. Подобные умозаключения можно сделать и в отношении нравственного
богословия. Возвращение к философии необходимо также в сфере постижения истин
веры, касающихся поведения верующих. Ввиду современных общественных,
экономических, политических и научных требований совесть человека теряет
ориентиры. В энциклике Veritatis splendor я писал, что многие проблемы
современного мира связаны с «кризисом понимания вопроса истины. Потеря
универсальной истины о добре, познаваемой человеческим разумом, неизбежно вела
и к изменению понятия совести: она уже не понимается в своем первичном
значении, т.е. как акт разумного познания личностью, призванной в определенной
ситуации использовать общее знание о добре и таким образом выражать свое
суждение о правильности формы поведения здесь и теперь. Появилась тенденция,
согласно которой совесть отдельной личности наделяется прерогативой автономно
определять крите-

рии добра и зла и соответственно выбирать линию поведения. Это представление
тесно связано с индивидуалистической этикой, утверждающей, что каждый человек
обладает собственной истиной, отличной от истины других людей»116 .

В вышеупомянутой энциклике я повсюду подчеркивал основную роль истины в сфере
нравственности. В контексте большинства неотложных этических проблем истина
требует от нравственного богословия внимательного размышления, которое сможет
четко указать ее корни, которые находятся в слове Божием. Чтобы осуществить эту
миссию, нравственное богословие должно использовать философскую этику, которая
обращается к истине о благе, а значит, не является ни субъективистской, ни
утилитаристской. Такой тип этики предполагает определенную философскую
антропологию и метафизику блага. Основываясь на этой цельной концепции, которая
очевидным образом связана с христианской святостью, а также с естественными и
сверхъестественными добродетелями, нравственное богословие сможет лучше и
эффективнее решать различные вопросы, входящие в его компетенцию: проблемы
мира, социальной справедливости, семьи, защиты жизни и охраны окружающей среды.

99. Богословские размышления в Церкви должны служить, прежде всего, проповеди
веры и катехизации117 . Проповедь, т.е. «керигма», должна призывать к
обращению, предлагая Христову истину, которая обретает кульминацию в Его
Пасхальной тайне, ибо только во Христе возможно познание полноты истины,
которая дарует спасение (см. Деян 4, 12; 1 Тим 2, 4-6).

В этом контексте становится понятно, почему такое важное значение имеет связь
философии не только с богословием, но и с катехизацией: последняя содержит
элементы философии, которые следует глубже постичь в свете веры. Учение,
заключенное в катехизации, оказывает воспитательное воздействие на человека.
Катехетическое учение, которое также является видом языкового сообщения, должно
в целостности представить аутентичное учение Церкви118 , показывая при этом и
его связь с жизнью верующих119 . Таким образом, осуществляется особое единство
учения и жизни, которое невозможно достичь иным путем, ибо в катехиза-

ции объектом передачи является не ряд понятийных истин, а тайна живого Бога120
.

С помощью философских размышлений можно объяснить связь истины с жизнью, связь
события с вероучительной истиной, особенно соотношение между трансцендентной
истиной и языком, понятным человеку121 . Взаимосвязь, которая возникает между
богословскими дисциплинами и концепциями, выработанными различными философскими
течениями, действительно может оказаться полезной для передачи веры и ее более
полного понимания.

Заключение

100. Прошло более ста лет с момента опубликования энциклики Льва XIII Aeterni
Patris, к которой я неоднократно обращалсяна этих страницах, и мне
представляется, что следует начать новые,более систематические размышления на
тему соотношения верыи философии. Очевидна роль философской мысли в развитии
культур и в формировании индивидуального и общественного поведения. Эта мысль
может оказывать существенное влияние, хотяи не всегда ощутимое непосредственно,
на богословие в целоми на его различные разделы. Поэтому мне показалось
правильными необходимым подчеркнуть значение философии с точки зренияпостижения
веры, а также показать ограничения, с которыми онасталкивается, когда забывает
об истинах Откровения или отвергаетих, ибо Церковь твердо убеждена, что вера и
разум «могут помогатьдруг другу»122 , выступая по отношению друг к другу в
качестве критического и очищающего фактора, а также стимула, побуждающего к
дальнейшим поискам и глубоким размышлениям.

101. Изучая историю человеческой мысли, особенно на Западе,можно легко найти в
ней богатство, служащее прогрессу человечества, которое возникло в результате
взаимообогащающей встречифилософии с богословием. Богословие, получив в дар
открытость идругие свойства, благодаря которым оно может существовать как

наука о вере, подготовило разум к принятию радикально новой вести, которая
заключена в Божественном Откровении. Это, несомненно, принесло пользу
философии, ибо благодаря этому она смогла увидеть новые перспективы и поставить
новые проблемы, которые должны все глубже исследоваться человеческим разумом.

В свете вышесказанного я считаю своей обязанностью подчеркнуть — подобно тому
как я отметил необходимость для богословия обретения соответствующей связи с
философией, — что и философия должна ради блага и прогресса человеческой мысли
восстановить связь с богословием. Она найдет в нем не размышления одного
человека, которые, даже будучи глубокими и богатыми, ограничены пределами и
особенностями отдельной личности, но богатство совместного размышления, ибо
естественной опорой богословия в его поисках истины является церковность123 , а
также традиция народа Божьего с ее богатством и многообразием знаний и культур,
связанных единством веры.

102. Подчеркивая, таким образом, значение философской мысли и ее истинные
границы, Церковь одновременно преследует двецели: защиту человеческого
достоинства и проповедь евангельского учения. Для осуществления этих задач в
наше время особеннонеобходимо помочь людям открыть их способность познавать
истину124 и обрести жажду глубочайшего и совершенного смысла существования. В
перспективе этих глубоких стремлений, вписанных Богом в природу человека,
становится более понятным такжечеловеческий и вместе с тем очеловечивающий
смысл слова Божьего. Благодаря посредничеству философии, которая стала
истинноймудростью, современный человек сможет убедиться, что чем больше он
открывается Христу, принимая Евангелие, тем человечнее онстановится.

103. Философия также является своеобразным зеркалом культуры народов.
Философия, которая развивается под воздействием требований богословия, сохраняя
гармонию с верой, является частьютой «евангелизации культуры», которую Павел VI
назвал одной изважнейших целей евангелизации125 .

Неустанно напоминая о необходимости новой евангелизации, я прошу философов все
глубже постигать истину, добро и красоту,

которые открывает нам слово Божие. Эта задача становится еще более неотложной,
если мы принимаем во внимание требования, которые выдвигает новое тысячелетие:
они относятся, прежде всего, к регионам и культурам с древней христианской
традицией. Это осмысление также следует признать важным и оригинальным вкладом
в дело новой евангелизации.

104. Часто философские размышления являются единственнойвозможностью найти
понимание и вступить в диалог с теми, ктоне исповедует нашу веру. Перемены,
которые происходят в современной философии, требуют внимательного и
компетентного участия верующих философов, которые способны заметить
ожидания,связанные с нынешним историческим моментом, новые интересыи проблемы.
Христианский философ, который выстраивает своюаргументацию в свете разума и в
соответствии с его правилами, хотя и руководствуется также высшими
соображениями, черпаемымив слове Божием, может строить умозаключения, доступные
и понятные в том числе тем, кто еще не видит полноты истины, заключенной в
Божественном Откровении. Возможность пониманияи диалога сегодня особенно
необходима потому, что неотложныепроблемы человечества — достаточно упомянуть
проблему экологии или проблемы мира и совместного существования различныхрас и
культур — могут быть решены благодаря открытому и честному сотрудничеству
христиан с приверженцами других религий,а также с теми, кто, не исповедуя
никакой религии, искренне желает обновления человечества. Это утверждали уже
Отцы ВторогоВатиканского Собора: «Желая такого диалога, который направлялся бы
лишь любовью к истине и с соблюдением, конечно, надлежащего благоразумия, мы, с
нашей стороны, готовы вести его с каждым: и с теми, кто пестует прекрасные
блага человеческого духа,хотя еще и не признает их Творца, и с теми, кто
противится Церкви и всячески ее преследует»126 . Философия, в которой сияет
хотябы только часть истины Христа — единственного окончательногоРазрешителя
проблем человека127 , — становится мощной опоройподлинной и одновременно
всеобщей этики, в которой сегодня такнуждается человечество.

105. В заключение этой энциклики я хочу в последний раз об-

ратиться с призывом, прежде всего, к богословам, посвятить особое внимание
философскому подтексту слова Божьего и начать размышления, которые во всей
полноте покажут теоретическую и практическую ценность богословских знаний. Я
хочу поблагодарить этих богословов за их служение в Церкви. Глубокая связь
богословской мудрости с философскими знаниями — это один из наиболее
оригинальных элементов наследия, который использует христианская традиция,
проникающая в тайну богооткровенной истины. Поэтому я призываю богословов
заново открыть и показать во всей глубине метафизический аспект истины и таким
образом начать критический и насущно необходимый диалог — как со всеми
течениями современной философской мысли, так и со всей философской традицией,
независимо от того, соответствует ли она слову Божьему или же ему противоречит.
Пусть они всегда помнят о предписании великого мыслителя и духовного учителя,
св. Бонавентуры, который призывал во вступлении к своему сочинению
«Путеводитель души к Богу»: «Пусть читатель не думает, что достаточно чтения
без духовного помазания, лицезрения — без благоволения, исследования — без
восторга, внимания — без духовной радости, усердия — без милосердия, знания —
без любви, ума — без смирения, учености — без Божественной благодати, умозрения
духовного зеркала — без мудрости, вдохновленной Богом»128 .

Я также призываю тех, кто отвечает за подготовку священников, как
академическую, так и пастырскую, прежде всего, заботиться о философской
подготовке тех, кто будет благовествовать Евангелие современному человеку, а в
первую очередь — тех, кто собирается посвятить себя изучению и преподаванию
богословия. Старайтесь выполнять свою работу в свете указаний II Ватиканского
Собора129 и последующих предписаний, указывающих на неотложную задачу, которая
стоит перед всеми нами и от которой никто не может отказаться, а именно,
обеспечить искреннюю и глубокую передачу истин веры. Нельзя забывать о важной
обязанности подобрать достойных профессоров, которые будут преподавать
философию в семинариях и других учебных заведениях Церкви130 . Необходимо,
чтобы это обучение базировалось на правильной научной подготовке,
систематически представляло великое наследие христиан-

ской традиции и вместе с тем учитывало актуальные нужды Церкви и мира.

106. Я также взываю к философам и преподавателям философии, чтобы они, следуя
всегда актуальной философской традиции, имели мужество вернуть философской
мысли аспект подлинной мудрости и истины, в том числе метафизической. Пусть они
будут готовы принять требования, которые выдвигает слово Божие, и постараются
ответить на них своими размышлениями и аргументацией. Пусть они всегда
стремятся к истине и чувствуют благо, которое в ней заключено. Благодаря этому
они смогут создать подлинную этику, в которой, особенно в настоящее время,
остро нуждается человечество. Церковь внимательно и благосклонно следит за их
исканиями; поэтому пусть они будут уверены, что она признает справедливую
автономию их науки. Я хочу призвать верующих, занимающихся философией, чтобы
они освещали различные сферы человеческой деятельности светочем разума, который
становится более надежным и ясным благодаря помощи веры.

Наконец, я хочу также обратиться к ученым, искания которых становятся для нас
источником все более полных знаний о мире в целом, о невероятном многообразии
его составных частей — как одушевленных, так и неодушевленных, — являющих
сложную атомную и молекулярную структуру. На этом поприще, особенно в нашем
веке, они достигли результатов, которым мы не перестаем удивляться. Я обращаю
также слова восхищения и призыва к тем первопроходцам науки, которым
человечество в значительной степени обязано нынешним уровнем своего развития,
но одновременно я также обязан призвать их продолжать исследования, никогда не
теряя из вида ту мудрость, которая наряду с достижениями науки и техники
рассматривает также философские и этические ценности, являющиеся характерным и
неотъемлемым проявлением сущности человеческой личности. Представители
естественных наук полностью осознают, что поиски истины — даже если она
касается лишь ограниченной части мира или человека — никогда не прекратятся;
они всегда устремлены к тому, что находится за пределами самого объекта
исследований, к вопросам, открывающим доступ к Тайне»131 .

107. Я прошу всех постараться увидеть внутренний мир человека, которого Христос
спас через тайну Своей любви, а также глубину его неустанных поисков истины и
смысла. Различные философские системы внушили ему ложное убеждение, что он
является абсолютным хозяином самого себя, может самостоятельно распоряжаться
своей судьбой и будущим, полагаться исключительно на себя и на собственные
силы. Величие человека никогда не осуществится таким образом. Оно осуществится
только тогда, когда человекпримет решение утвердиться в истине, воздвигнуть
свой дом под сенью Мудрости и обитать в нем. Только пребывая в истине,
человексможет полностью понять смысл своей свободы и своего призвания любить и
познавать Бога, и в осуществлении этого призванияон полностью обретет самого
себя.

108. Теперь я хочу обратиться мыслями к Той, Которую Церковьв своих молитвах
именует Престолом Мудрости. Ее жизнь — словно притча, способная пролить свет на
размышления, которыея здесь представил, ибо можно заметить глубокое сходство
призвания Блаженной Девы с призванием подлинной философии. Подобно тому как
Дева Мария была призвана дать Свое человеческоеи женское естество, чтобы Слово
Божие могло обрести плотьи стать одним из нас, так и философия должна своими
рациональными и критическими размышлениями способствовать тому, чтобы
богословие как осознание веры было плодотворным и эффективным. Как и Мария,
Которая, выразив Свое согласие на замысел,о котором Ей возвестил архангел
Гавриил, не потеряла Своего человеческого достоинства и свободы, так и
философия, принимаятребования, которые предъявляет евангельская истина, не
теряетавтономию, ибо благодаря этому все ее поиски направлены к возвышенным
целям. Это хорошо понимали в древности праведныехристианские монахи, которые
называли Марию «мысленной трапезой веры»132 . Они видели в Ней правдивое
отображение истинной философии и были убеждены, что необходимо философствовать
с Марией.

Пусть Престол Мудрости станет безопасной гаванью для тех, кто посвятил свою
жизнь поискам истины. Остается пожелать, чтобы путь к мудрости, конечной и
подлинной цели каждой науки,

был очищен от всяких препятствий благодаря заступничеству Той, Которая, родив
Истину и сохраняя истину в сердце Своем, навеки передала ее всему человечеству.

Дано в Риме, у Св. Петра, 14 сентября 1998 г.,

в праздник Воздвижения Святого Креста Господня, в двадцатый год моего
Понтификата

Примечания


2 
3 
4 
5 
6 
7 
8 
9 
10 
11 
12 
13 
14 
15 
16 
17 
18 
19 
20 
21 
22 
23 
24 
25 
26 
27 
28 
29 
30 
31 
32 
33 
34 
35 
36 
37 
38 
39 
40 
41 
42 
43 
44 
45 
46 
47 
48 
49 
50 
51 
52 
53 
54 
55 
56 
57 
58 
59 
60 
61 Cp. Пий IX, бреве Eximiam tuam (15 июня 1857): DS 2828-2831; бревеGravissimus inter (11 декабря 1862): DS 2850-2861.
62 Cp. Священная Конгрегация вероучения, Декрет Errores ontologistarum (18сентября 1861 г.): DS 2841-2847.
63 См. I Ватиканский Собор, Догматическая конституция о католической вере Dei Filius, глава II: DS 3004; и кан. 2, 1: DS 3026.
64 Там же, глава IV: DS 3015; цит. также в: II Ватиканский Собор, Пастырскаяконституция Gaudium et spes, 59.
65 I Ватиканский Собор, Догматическая конституция о католической вере DeiFilius, глава IV: DS 3017.
66 См. энциклика Pascendi dominici gregis (8 сентября 1907 г.): AAS 40 (1907),596-597.
67 См. Пий XI, энциклика Divini Redemptoris (19 марта 1937 г.): AAS 29 (1937),65-106.
68 Энциклика Humani generis (12 августа 1950): AAS 42 (1950), 592-563.
69 Там же, l.c., с. 563-564.
70 См. Иоанн Павел II, Апостольская конституция Pastor Bonus (28 июня1988), пп. 48–49: AAS 80 (1988), 873; Священная Конгрегация вероучения, Инструкция о призвании богослова в Церкви Donum veritatis (24 мая 1990), 18:AAS 82 (1990), 1558.
71 См. Инструкцию о некоторых аспектах «теологии освобождения» Libertatisnuntius (6 августа 1984 г.), VII-X: AAS 76 (1984), 890-903.
72 Уже Отцы I Ватиканского Собора властно и недвусмысленно осудили это заблуждение, заявив: «Что касается веры (...), согласно учению КатолическойЦеркви, она является сверхъестественной добродетелью, благодаря которойпод влиянием Божественного вдохновения и благодати мы признаем правдивыми истины, которые Он открывает, не потому, что в свете естественного разума мы постигаем внутреннюю истину явлений, а по причине авторитета Самого Бога, Который дал Откровение и не может ни заблуждаться, ни вводитьв заблуждение» (Догматическая Конституция Dei Filius, глава III: DS 3008, и кан.3,2: DS 3032). С другой стороны, Собор утверждал, что разум «не способен постичь эти тайны как истины, свойственные его познанию» (там же, глава IV:DS 3016). На этом основании Собор сделал практический вывод: «Верующиехристиане не только не имеют права защищать как доказанные научные достижения,
взгляды, объявленные противоречащими доктрине веры, особенно если они осуждены Церковью, но и обязаны относиться к ним как к заблуждениям, которые носят лишь видимость правды» (там же, глава IV: DS 3018).
73 См. пп. 9-10.
74 Там же, п. 10.
75 См. там же, п. 21.
76 См. там же, п. 10.
77 См. энциклика Humani generis (12 августа 1950): AAS 42 (1950), 565-567;571-573.
78 См. энциклика Aeterni Patris (4 августа 1879): AAS 11 (1878-1879), 97-115.
79 Там же, l.c., 109.
80 См. пп. 14-15.
81 См. там же, 20-21.
82 Там же, 22; см. Иоанн Павел II, энциклика Redemptoris hominis (4 марта1979 г.), 8: AAS 71 (1979), 271-272.
83 Декрет о подготовке к священству Optatam totius, 15.
84 См. Иоанн Павел II, Апостольская конституция Sapientia christiana (15 апреля 1979), с. 79-80: AAS 71 (1979), 495–496; Постсинодальное Апостольское обращение Pastores dabo vobis (25 марта 1992 г.), 52: AAS 84 (1992) 750-751; см. также отдельные комментарии философии св. Фомы: Выступление в ПапскомУниверситете в «Angelicum» (17 ноября 1979 г.): Insegnamenti II, 2 (1979),1177-1189; Выступление на VIII Международном конгрессе томистов (13 сентября 1980 г.): Insegnamenti III, 2 (1980), 604-615; Выступление на Международном конгрессе общества «Сан Томмазо» «Душа в учении св. Фомы» (4 февраля 1986 г.): Insegnamenti IX, 1 (1986), 18-24. Кроме того, Священная Конгрегация католического воспитания, Ratio fundamentalis institutionis sacerdotalis (6января 1970 г.), 70-75: AAS 62 (1970), 366-368; Декрет Sacra Theologia (20 января 1972 г.): AAS 64 (1972), 583-586.
85 См. Пастырская конституция о Церкви в современном мире Gaudium etspes, 57; 62.
86 См. там же, 44.
87 См. V Латеранский Собор, булла Apostolici regiminis sollicitudo, Сессия VIII;Conc. Oecum. Decreta, 1991, 605-606.
88 См. II Ватиканский Собор, Догматическая конституция о БожественномОткровении Dei Verbum, 10.
89 Св. Фома Аквинский, Summa Theologiae, II-II, 5, 3 ad 2.
90 «Для фундаментального богословия изучение ситуаций, в которых человек задается извечными вопросами о смысле жизни, о цели, которую он хочет ей придать, и о том, что его ожидает после смерти, является необходимой подготовкой, чтобы и в наши дни вера могла во всей полноте указатьпуть разуму, искренне ищущему истину». Иоанн Павел II, Письмо к участникам Международного конгресса по фундаментальному богословию в 125годовщину «Dei Filius» (30 сентября 1995 г.), 4: «L'Osservatore Romano» ( 3 октября 1995 ), с. 8.
91 Там же.
92 См. II Ватиканский Собор, Пастырская конституция о Церкви в современном мире Gaudium et spes, 15; Декрет о миссионерской деятельности ЦерквиAd gentes, 22.
93 Св. Фома Аквинский, De Caelo, 1, 22.
94 См. Второй Ватиканский Собор, Пастырская конституция о Церкви в современном мире Gaudium et spes, 53-59.
95 Св. Августин, De praedestinatione sanctorum, 2, 5: PL 44, 963.
96 Св. Августин, De fide, spe et caritate, 7: CCL 64, 61.
97 См. Халкидонский Собор, Symbolum, Definitio: DS 302.
98 См. Иоанн Павел II, энциклика Redemptor hominis (4 марта 1979 г.), 15: AAS71 (1979), 286-289; с. 26-29.
99 См., например, Фома Аквинский, Summa Theologiae, I,16, 1; Бонавентура,Coll. In Hex., 3, 8, 1.
100 Пастырская конституция о Церкви в современном мире Gaudium et spes, 15.
101 См. Иоанн Павел II, энциклика Veritatis splendor (6 августа 1993 г.), 57-61: AAS 85 (1993), 1179-1182.
102 См. I Ватиканский Собор, Догматическая конституция о католической вере Dei Filius, глава IV: DS 3016.
103 См. IV Латеранский Собор, De errore abbatis Ioachim, II: DS 806.
104 См. II Ватиканский Собор, Догматическая конституция о БожественномОткровении Dei Verbum, 24; Декрет о подготовке к священству, Optatam totius,16.
105 См. Иоанн Павел II, энциклика Evangelium vitae (25 марта 1995 г.) AAS 87(1995), 481.
106 Так я писал в моей первой энциклике, комментируя слова из Евангелия отИоанна «И познаете истину, и истина сделает вас свободными» (Ин 8, 32): «Вэтих словах заключено следующее фундаментальное требование и одновременно предостережение: требование честного отношения к истине как к условию подлинной свободы; предостережение от видимости свободы, от всякой поверхностной и односторонней свободы, не проникающей в глубинуистины о человеке и мире. Для нас Христос и сегодня, через две тысячи лет,Тот, Кто несет человеку свободу, основанную на истине, Тот, Кто освобождает человека от всего, что его свободу ограничивает, уменьшает, подрывает усамого корня, Кто освобождает человека от всего, что отрывает дух человекаот свободы, делает несвободным его ум и сердце». Redemptor hominis (4 марта 1979 г.), 12: AAS 71 (1979), 280-281; c. 22.
107 Речь на открытии Собора (11 октября 1962 г.): AAS 54 (1962), 792.
108 Священная Конгрегация вероучения, Инструкция о призвании богословав Церкви Donum veritatis (24 мая 1990), 7-8: AAS 82 (1990), 1552-1553.
109 Комментируя слова евангелиста Иоанна 16, 12-13, я написал в энцикликеDominum et vivificantem: «Иисус характеризует Утешителя — Духа истиныкак Того, Кто «научит вас всему и напомнит вам все», как Того, Кто будет «свидетельствовать» о Нем. Далее говорится: «Наставит вас на всякую истину». Это «наставление на всякую истину» в том, чего апостолы «теперь не могут вместить», представляется прежде всего необходимым ввиду уничижения Христа через страдания и смерть на кресте, которая в то время, когда он произносил эти слова, была так близко. Однако в дальнейшем это «наставление на всякую истину» кажется необходимым не только в связи с самим соблазном креста, но и в связи со всем, что Христос делал и чему учил (Деян 1, 1). Действительно, тайна Христа, взятая в целом, требует веры, ибо лишь вера подобающим образом вводит человека в реальность богооткровенной тайны. Итак, «наставление на всякую истину» совершается в вере и через веру, что является деянием Духа истины и результатом Его действий в человеке. Святой Дух должен стать высшим наставником человека, светочем для человеческого духа» (п. 6: AAS 78 (1986), 815-816).
110 См. II Ватиканский Собор, Догматическая конституция о БожественномОткровении Dei Verbum, 13.
111 См. Папская библейская комиссия, Инструкция об исторической истинеЕвангелия (21 апреля 1964): AAS 56 (1964), 713.
112 «Очевидно, что Церковь не может быть связана ни с одной скоротечнойфилософской системой: понятия и термины, которые католические мыслители, пользующиеся всеобщим признанием, выработали в течение многих веков, чтобы в какой-то степени постичь и понять догматы, не покоятся настоль недолговечном фундаменте. Они находят опору в принципах и понятиях, сформулированных истинным познанием сущего; получая эти знания,человеческий разум следовал за богооткровенной истиной, которая благодаря Церкви освещала его словно звезда. Следовательно, неудивительно, что некоторые из этих понятий не только были использованы на ВселенскихСоборах, но получили такое
решительное одобрение, что мы не можем отказаться от них». Энциклика Humani generis (12 августа 1950 г.): AAS 42 (1950),566-567; cp. Международная богословская комиссия, Документ Interpretationisproblema (октябрь 1950 г.) Ench. Vat., №№ 2717-2811.
113 «Само же значение догматических формулировок всегда остается в Церквиистинным и постоянным, даже если оно глубже объясняется и более полнопонимается. Поэтому верующие должны отвергнуть мнение, (...) что догматические формулировки (или их разновидности) не могут выражать истину точно, а лишь в приблизительных выражениях, которые искажают или изменяют ее». Священная Конгрегация вероучения, Декларация о Церкви в защитукатолической доктрины против некоторых современных заблуждений,Mysterium Ecclesiae (24 июня 1973 г.), 5: AAS 65 (1973), 403.
114 См. Священная Конгрегация вероучения, декрет Lamentabili (3 июля 1907),26: AAS 40 (1907), 473.
115 См. Иоанн Павел II, Выступление в Папском Университете «Angelicum» (17ноября 1979 г.) 6, Insegnamenti, II, 2 (1979), 1183-1185.
116 П. 32: AAS 85 (1993), 1159-1160.
117 См. Иоанн Павел II, Апостольское обращение Catechesi tradendae (Общиеуказания по катехизации) 16 октября 1979 г., 30: AAS, 71 (1979), 1302-1303; Священная Конгрегация вероучения, Инструкция о призвании богослова в Церкви (24 мая 1990 г.), 7: AAS 82 (1990), 1552-1553.
118 См. Иоанн Павел II, Апостольское обращение Catechesi tradendae 16 октября 1979 г., 30: AAS 71 (1979), 1302-1303.
119 См. там же, 22, l.c., 1295-1296.
120 См. там же, 7, l.c., 1282.
121 См. там же, 59, l.c., 1325.
122 I Ватиканский Собор, Догматическая Конституция о католической вереDei Filius, глава IV: DS 3019.
123 «Никто не имеет права сводить богословие к простому изложению собственных идей. Но каждый должен быть в тесном, сознательном единстве с миссией преподавания истины, за которую отвечает Церковь». Иоанн Павел II,энциклика Redemptor hominis (4 марта 1979 г.), 19: AAS 71 (1979), 308.
124 См. II Ватиканский Собор, Декларация о религиозной свободе Dignitatishumanae, 1-3.
125 См. aпостольское обращение Evangelii nuntiandi (8 декабря 1975 г.), 20: AAS68 (1976), 18-19.
126 Пастырская конституция о Церкви в современном мире Gaudium et spes,92.
127 См. там же, 10.
128 Prologus, 4: Opera omnia, Флоренция 1891, т. V, 296; см. также: Бонавентура, Путеводитель души к Богу, Москва 1993, с. 45–47.
129 См. Декрет о подготовке к священству Optatam totius, 15.
130 См. Иоанн Павел II, Апостольская конституция Sapientia Christiana (15 апреля 1979 г.), ст. 67-68: AAS 71 (1979), 491–492.
131 Иоанн Павел II, Выступление по случаю 600-летия богословского факультета Ягеллонского университета (8 июня 1997 г.): «L'Osservatore Romano», 9-10июня 1997, с. 12.
132 «He noera tes pisteos trapeza» (св. пифаний Кипрский, Homilia in laudesSanctae Mariae Deiparae: PG 43, 493).

\end{document}
